\input texinfo   @c -*-texinfo-*-

@c %**start of header
@setfilename wget.info
@settitle GNU Wget Manual
@c Disable the monstrous rectangles beside overfull hbox-es.
@finalout
@c Use `odd' to print double-sided.
@setchapternewpage on
@c %**end of header

@iftex
@c Remove this if you don't use A4 paper.
@afourpaper
@end iftex

@c This should really be generated automatically, possibly by including
@c an auto-generated file.
@set VERSION 1.8.2
@set UPDATED May 2002

@dircategory Net Utilities
@dircategory World Wide Web
@direntry
* Wget: (wget).         The non-interactive network downloader.
@end direntry

@ifinfo
This file documents the the GNU Wget utility for downloading network
data.

@c man begin COPYRIGHT
Copyright @copyright{} 1996, 1997, 1998, 2000, 2001 Free Software
Foundation, Inc.

Permission is granted to make and distribute verbatim copies of
this manual provided the copyright notice and this permission notice
are preserved on all copies.

@ignore
Permission is granted to process this file through TeX and print the
results, provided the printed document carries a copying permission
notice identical to this one except for the removal of this paragraph
(this paragraph not being relevant to the printed manual).
@end ignore
Permission is granted to copy, distribute and/or modify this document
under the terms of the GNU Free Documentation License, Version 1.1 or
any later version published by the Free Software Foundation; with the
Invariant Sections being ``GNU General Public License'' and ``GNU Free
Documentation License'', with no Front-Cover Texts, and with no
Back-Cover Texts.  A copy of the license is included in the section
entitled ``GNU Free Documentation License''.
@c man end
@end ifinfo

@titlepage
@title GNU Wget
@subtitle The noninteractive downloading utility
@subtitle Updated for Wget @value{VERSION}, @value{UPDATED}
@author by Hrvoje Nik@v{s}i@'{c} and the developers

@ignore
@c man begin AUTHOR
Originally written by Hrvoje Niksic <hniksic@arsdigita.com>.
@c man end
@c man begin SEEALSO
GNU Info entry for @file{wget}.
@c man end
@end ignore

@page
@vskip 0pt plus 1filll
Copyright @copyright{} 1996, 1997, 1998, 2000, 2001 Free Software
Foundation, Inc.

Permission is granted to copy, distribute and/or modify this document
under the terms of the GNU Free Documentation License, Version 1.1 or
any later version published by the Free Software Foundation; with the
Invariant Sections being ``GNU General Public License'' and ``GNU Free
Documentation License'', with no Front-Cover Texts, and with no
Back-Cover Texts.  A copy of the license is included in the section
entitled ``GNU Free Documentation License''.
@end titlepage

@ifinfo
@node Top, Overview, (dir), (dir)
@top Wget @value{VERSION}

This manual documents version @value{VERSION} of GNU Wget, the freely
available utility for network download.

Copyright @copyright{} 1996, 1997, 1998, 2000, 2001 Free Software
Foundation, Inc.

@menu
* Overview::            Features of Wget.
* Invoking::            Wget command-line arguments.
* Recursive Retrieval:: Description of recursive retrieval.
* Following Links::     The available methods of chasing links.
* Time-Stamping::       Mirroring according to time-stamps.
* Startup File::        Wget's initialization file.
* Examples::            Examples of usage.
* Various::             The stuff that doesn't fit anywhere else.
* Appendices::          Some useful references.
* Copying::             You may give out copies of Wget and of this manual.
* Concept Index::       Topics covered by this manual.
@end menu
@end ifinfo

@node Overview, Invoking, Top, Top
@chapter Overview
@cindex overview
@cindex features

@c man begin DESCRIPTION
GNU Wget is a free utility for non-interactive download of files from
the Web.  It supports @sc{http}, @sc{https}, and @sc{ftp} protocols, as
well as retrieval through @sc{http} proxies.

@c man end
This chapter is a partial overview of Wget's features.

@itemize @bullet
@item
@c man begin DESCRIPTION
Wget is non-interactive, meaning that it can work in the background,
while the user is not logged on.  This allows you to start a retrieval
and disconnect from the system, letting Wget finish the work.  By
contrast, most of the Web browsers require constant user's presence,
which can be a great hindrance when transferring a lot of data.
@c man end

@sp 1
@item
@ignore
@c man begin DESCRIPTION

@c man end
@end ignore
@c man begin DESCRIPTION
Wget can follow links in @sc{html} pages and create local versions of
remote web sites, fully recreating the directory structure of the
original site.  This is sometimes referred to as ``recursive
downloading.''  While doing that, Wget respects the Robot Exclusion
Standard (@file{/robots.txt}).  Wget can be instructed to convert the
links in downloaded @sc{html} files to the local files for offline
viewing.
@c man end

@sp 1
@item
File name wildcard matching and recursive mirroring of directories are
available when retrieving via @sc{ftp}.  Wget can read the time-stamp
information given by both @sc{http} and @sc{ftp} servers, and store it
locally.  Thus Wget can see if the remote file has changed since last
retrieval, and automatically retrieve the new version if it has.  This
makes Wget suitable for mirroring of @sc{ftp} sites, as well as home
pages.

@sp 1
@item
@ignore
@c man begin DESCRIPTION

@c man end
@end ignore
@c man begin DESCRIPTION
Wget has been designed for robustness over slow or unstable network
connections; if a download fails due to a network problem, it will
keep retrying until the whole file has been retrieved.  If the server
supports regetting, it will instruct the server to continue the
download from where it left off.
@c man end

@sp 1
@item
Wget supports proxy servers, which can lighten the network load, speed
up retrieval and provide access behind firewalls.  However, if you are
behind a firewall that requires that you use a socks style gateway, you
can get the socks library and build Wget with support for socks.  Wget
also supports the passive @sc{ftp} downloading as an option.

@sp 1
@item
Builtin features offer mechanisms to tune which links you wish to follow
(@pxref{Following Links}).

@sp 1
@item
The retrieval is conveniently traced with printing dots, each dot
representing a fixed amount of data received (1KB by default).  These
representations can be customized to your preferences.

@sp 1
@item
Most of the features are fully configurable, either through command line
options, or via the initialization file @file{.wgetrc} (@pxref{Startup
File}).  Wget allows you to define @dfn{global} startup files
(@file{/usr/local/etc/wgetrc} by default) for site settings.

@ignore
@c man begin FILES
@table @samp
@item /usr/local/etc/wgetrc
Default location of the @dfn{global} startup file.

@item .wgetrc
User startup file.
@end table
@c man end
@end ignore

@sp 1
@item
Finally, GNU Wget is free software.  This means that everyone may use
it, redistribute it and/or modify it under the terms of the GNU General
Public License, as published by the Free Software Foundation
(@pxref{Copying}).
@end itemize

@node Invoking, Recursive Retrieval, Overview, Top
@chapter Invoking
@cindex invoking
@cindex command line
@cindex arguments
@cindex nohup

By default, Wget is very simple to invoke.  The basic syntax is:

@example
@c man begin SYNOPSIS
wget [@var{option}]@dots{} [@var{URL}]@dots{}
@c man end
@end example

Wget will simply download all the @sc{url}s specified on the command
line.  @var{URL} is a @dfn{Uniform Resource Locator}, as defined below.

However, you may wish to change some of the default parameters of
Wget.  You can do it two ways: permanently, adding the appropriate
command to @file{.wgetrc} (@pxref{Startup File}), or specifying it on
the command line.

@menu
* URL Format::
* Option Syntax::
* Basic Startup Options::
* Logging and Input File Options::
* Download Options::
* Directory Options::
* HTTP Options::
* FTP Options::
* Recursive Retrieval Options::
* Recursive Accept/Reject Options::
@end menu

@node URL Format, Option Syntax, Invoking, Invoking
@section URL Format
@cindex URL
@cindex URL syntax

@dfn{URL} is an acronym for Uniform Resource Locator.  A uniform
resource locator is a compact string representation for a resource
available via the Internet.  Wget recognizes the @sc{url} syntax as per
@sc{rfc1738}.  This is the most widely used form (square brackets denote
optional parts):

@example
http://host[:port]/directory/file
ftp://host[:port]/directory/file
@end example

You can also encode your username and password within a @sc{url}:

@example
ftp://user:password@@host/path
http://user:password@@host/path
@end example

Either @var{user} or @var{password}, or both, may be left out.  If you
leave out either the @sc{http} username or password, no authentication
will be sent.  If you leave out the @sc{ftp} username, @samp{anonymous}
will be used.  If you leave out the @sc{ftp} password, your email
address will be supplied as a default password.@footnote{If you have a
@file{.netrc} file in your home directory, password will also be
searched for there.}

@strong{Important Note}: if you specify a password-containing @sc{url}
on the command line, the username and password will be plainly visible
to all users on the system, by way of @code{ps}.  On multi-user systems,
this is a big security risk.  To work around it, use @code{wget -i -}
and feed the @sc{url}s to Wget's standard input, each on a separate
line, terminated by @kbd{C-d}.

You can encode unsafe characters in a @sc{url} as @samp (quoted as
@samp{%25}), @samp{:} (quoted as @samp{%3A}), and @samp{@@} (quoted as
@samp{%40}).  Refer to @sc{rfc1738} for a comprehensive list of unsafe
characters.

Wget also supports the @code{type} feature for @sc{ftp} @sc{url}s.  By
default, @sc{ftp} documents are retrieved in the binary mode (type
@samp{i}), which means that they are downloaded unchanged.  Another
useful mode is the @samp{a} (@dfn{ASCII}) mode, which converts the line
delimiters between the different operating systems, and is thus useful
for text files.  Here is an example:

@example
ftp://host/directory/file;type=a
@end example

Two alternative variants of @sc{url} specification are also supported,
because of historical (hysterical?) reasons and their widespreaded use.

@sc{ftp}-only syntax (supported by @code{NcFTP}):
@example
host:/dir/file
@end example

@sc{http}-only syntax (introduced by @code{Netscape}):
@example
host[:port]/dir/file
@end example

These two alternative forms are deprecated, and may cease being
supported in the future.

If you do not understand the difference between these notations, or do
not know which one to use, just use the plain ordinary format you use
with your favorite browser, like @code{Lynx} or @code{Netscape}.

@node Option Syntax, Basic Startup Options, URL Format, Invoking
@section Option Syntax
@cindex option syntax
@cindex syntax of options

Since Wget uses GNU getopts to process its arguments, every option has a
short form and a long form.  Long options are more convenient to
remember, but take time to type.  You may freely mix different option
styles, or specify options after the command-line arguments.  Thus you
may write:

@example
wget -r --tries=10 http://fly.srk.fer.hr/ -o log
@end example

The space between the option accepting an argument and the argument may
be omitted.  Instead @samp{-o log} you can write @samp{-olog}.

You may put several options that do not require arguments together,
like:

@example
wget -drc @var{URL}
@end example

This is a complete equivalent of:

@example
wget -d -r -c @var{URL}
@end example

Since the options can be specified after the arguments, you may
terminate them with @samp{--}.  So the following will try to download
@sc{url} @samp{-x}, reporting failure to @file{log}:

@example
wget -o log -- -x
@end example

The options that accept comma-separated lists all respect the convention
that specifying an empty list clears its value.  This can be useful to
clear the @file{.wgetrc} settings.  For instance, if your @file{.wgetrc}
sets @code{exclude_directories} to @file{/cgi-bin}, the following
example will first reset it, and then set it to exclude @file{/~nobody}
and @file{/~somebody}.  You can also clear the lists in @file{.wgetrc}
(@pxref{Wgetrc Syntax}).

@example
wget -X '' -X /~nobody,/~somebody
@end example

@c man begin OPTIONS

@node Basic Startup Options, Logging and Input File Options, Option Syntax, Invoking
@section Basic Startup Options

@table @samp
@item -V
@itemx --version
Display the version of Wget.

@item -h
@itemx --help
Print a help message describing all of Wget's command-line options.

@item -b
@itemx --background
Go to background immediately after startup.  If no output file is
specified via the @samp{-o}, output is redirected to @file{wget-log}.

@cindex execute wgetrc command
@item -e @var{command}
@itemx --execute @var{command}
Execute @var{command} as if it were a part of @file{.wgetrc}
(@pxref{Startup File}).  A command thus invoked will be executed
@emph{after} the commands in @file{.wgetrc}, thus taking precedence over
them.
@end table

@node Logging and Input File Options, Download Options, Basic Startup Options, Invoking
@section Logging and Input File Options

@table @samp
@cindex output file
@cindex log file
@item -o @var{logfile}
@itemx --output-file=@var{logfile}
Log all messages to @var{logfile}.  The messages are normally reported
to standard error.

@cindex append to log
@item -a @var{logfile}
@itemx --append-output=@var{logfile}
Append to @var{logfile}.  This is the same as @samp{-o}, only it appends
to @var{logfile} instead of overwriting the old log file.  If
@var{logfile} does not exist, a new file is created.

@cindex debug
@item -d
@itemx --debug
Turn on debug output, meaning various information important to the
developers of Wget if it does not work properly.  Your system
administrator may have chosen to compile Wget without debug support, in
which case @samp{-d} will not work.  Please note that compiling with
debug support is always safe---Wget compiled with the debug support will
@emph{not} print any debug info unless requested with @samp{-d}.
@xref{Reporting Bugs}, for more information on how to use @samp{-d} for
sending bug reports.

@cindex quiet
@item -q
@itemx --quiet
Turn off Wget's output.

@cindex verbose
@item -v
@itemx --verbose
Turn on verbose output, with all the available data.  The default output
is verbose.

@item -nv
@itemx --non-verbose
Non-verbose output---turn off verbose without being completely quiet
(use @samp{-q} for that), which means that error messages and basic
information still get printed.

@cindex input-file
@item -i @var{file}
@itemx --input-file=@var{file}
Read @sc{url}s from @var{file}, in which case no @sc{url}s need to be on
the command line.  If there are @sc{url}s both on the command line and
in an input file, those on the command lines will be the first ones to
be retrieved.  The @var{file} need not be an @sc{html} document (but no
harm if it is)---it is enough if the @sc{url}s are just listed
sequentially.

However, if you specify @samp{--force-html}, the document will be
regarded as @samp{html}.  In that case you may have problems with
relative links, which you can solve either by adding @code{<base
href="@var{url}">} to the documents or by specifying
@samp{--base=@var{url}} on the command line.

@cindex force html
@item -F
@itemx --force-html
When input is read from a file, force it to be treated as an @sc{html}
file.  This enables you to retrieve relative links from existing
@sc{html} files on your local disk, by adding @code{<base
href="@var{url}">} to @sc{html}, or using the @samp{--base} command-line
option.

@cindex base for relative links in input file
@item -B @var{URL}
@itemx --base=@var{URL}
When used in conjunction with @samp{-F}, prepends @var{URL} to relative
links in the file specified by @samp{-i}.
@end table

@node Download Options, Directory Options, Logging and Input File Options, Invoking
@section Download Options

@table @samp
@cindex bind() address
@cindex client IP address
@cindex IP address, client
@item --bind-address=@var{ADDRESS}
When making client TCP/IP connections, @code{bind()} to @var{ADDRESS} on
the local machine.  @var{ADDRESS} may be specified as a hostname or IP
address.  This option can be useful if your machine is bound to multiple
IPs.

@cindex retries
@cindex tries
@cindex number of retries
@item -t @var{number}
@itemx --tries=@var{number}
Set number of retries to @var{number}.  Specify 0 or @samp{inf} for
infinite retrying.

@item -O @var{file}
@itemx --output-document=@var{file}
The documents will not be written to the appropriate files, but all will
be concatenated together and written to @var{file}.  If @var{file}
already exists, it will be overwritten.  If the @var{file} is @samp{-},
the documents will be written to standard output.  Including this option
automatically sets the number of tries to 1.

@cindex clobbering, file
@cindex downloading multiple times
@cindex no-clobber
@item -nc
@itemx --no-clobber
If a file is downloaded more than once in the same directory, Wget's
behavior depends on a few options, including @samp{-nc}.  In certain
cases, the local file will be @dfn{clobbered}, or overwritten, upon
repeated download.  In other cases it will be preserved.

When running Wget without @samp{-N}, @samp{-nc}, or @samp{-r},
downloading the same file in the same directory will result in the
original copy of @var{file} being preserved and the second copy being
named @samp{@var{file}.1}.  If that file is downloaded yet again, the
third copy will be named @samp{@var{file}.2}, and so on.  When
@samp{-nc} is specified, this behavior is suppressed, and Wget will
refuse to download newer copies of @samp{@var{file}}.  Therefore,
``@code{no-clobber}'' is actually a misnomer in this mode---it's not
clobbering that's prevented (as the numeric suffixes were already
preventing clobbering), but rather the multiple version saving that's
prevented.

When running Wget with @samp{-r}, but without @samp{-N} or @samp{-nc},
re-downloading a file will result in the new copy simply overwriting the
old.  Adding @samp{-nc} will prevent this behavior, instead causing the
original version to be preserved and any newer copies on the server to
be ignored.

When running Wget with @samp{-N}, with or without @samp{-r}, the
decision as to whether or not to download a newer copy of a file depends
on the local and remote timestamp and size of the file
(@pxref{Time-Stamping}).  @samp{-nc} may not be specified at the same
time as @samp{-N}.

Note that when @samp{-nc} is specified, files with the suffixes
@samp{.html} or (yuck) @samp{.htm} will be loaded from the local disk
and parsed as if they had been retrieved from the Web.

@cindex continue retrieval
@cindex incomplete downloads
@cindex resume download
@item -c
@itemx --continue
Continue getting a partially-downloaded file.  This is useful when you
want to finish up a download started by a previous instance of Wget, or
by another program.  For instance:

@example
wget -c ftp://sunsite.doc.ic.ac.uk/ls-lR.Z
@end example

If there is a file named @file{ls-lR.Z} in the current directory, Wget
will assume that it is the first portion of the remote file, and will
ask the server to continue the retrieval from an offset equal to the
length of the local file.

Note that you don't need to specify this option if you just want the
current invocation of Wget to retry downloading a file should the
connection be lost midway through.  This is the default behavior.
@samp{-c} only affects resumption of downloads started @emph{prior} to
this invocation of Wget, and whose local files are still sitting around.

Without @samp{-c}, the previous example would just download the remote
file to @file{ls-lR.Z.1}, leaving the truncated @file{ls-lR.Z} file
alone.

Beginning with Wget 1.7, if you use @samp{-c} on a non-empty file, and
it turns out that the server does not support continued downloading,
Wget will refuse to start the download from scratch, which would
effectively ruin existing contents.  If you really want the download to
start from scratch, remove the file.

Also beginning with Wget 1.7, if you use @samp{-c} on a file which is of
equal size as the one on the server, Wget will refuse to download the
file and print an explanatory message.  The same happens when the file
is smaller on the server than locally (presumably because it was changed
on the server since your last download attempt)---because ``continuing''
is not meaningful, no download occurs.

On the other side of the coin, while using @samp{-c}, any file that's
bigger on the server than locally will be considered an incomplete
download and only @code{(length(remote) - length(local))} bytes will be
downloaded and tacked onto the end of the local file.  This behavior can
be desirable in certain cases---for instance, you can use @samp{wget -c}
to download just the new portion that's been appended to a data
collection or log file.

However, if the file is bigger on the server because it's been
@emph{changed}, as opposed to just @emph{appended} to, you'll end up
with a garbled file.  Wget has no way of verifying that the local file
is really a valid prefix of the remote file.  You need to be especially
careful of this when using @samp{-c} in conjunction with @samp{-r},
since every file will be considered as an "incomplete download" candidate.

Another instance where you'll get a garbled file if you try to use
@samp{-c} is if you have a lame @sc{http} proxy that inserts a
``transfer interrupted'' string into the local file.  In the future a
``rollback'' option may be added to deal with this case.

Note that @samp{-c} only works with @sc{ftp} servers and with @sc{http}
servers that support the @code{Range} header.

@cindex progress indicator
@cindex dot style
@item --progress=@var{type}
Select the type of the progress indicator you wish to use.  Legal
indicators are ``dot'' and ``bar''.

The ``bar'' indicator is used by default.  It draws an ASCII progress
bar graphics (a.k.a ``thermometer'' display) indicating the status of
retrieval.  If the output is not a TTY, the ``dot'' bar will be used by
default.

Use @samp{--progress=dot} to switch to the ``dot'' display.  It traces
the retrieval by printing dots on the screen, each dot representing a
fixed amount of downloaded data.

When using the dotted retrieval, you may also set the @dfn{style} by
specifying the type as @samp{dot:@var{style}}.  Different styles assign
different meaning to one dot.  With the @code{default} style each dot
represents 1K, there are ten dots in a cluster and 50 dots in a line.
The @code{binary} style has a more ``computer''-like orientation---8K
dots, 16-dots clusters and 48 dots per line (which makes for 384K
lines).  The @code{mega} style is suitable for downloading very large
files---each dot represents 64K retrieved, there are eight dots in a
cluster, and 48 dots on each line (so each line contains 3M).

Note that you can set the default style using the @code{progress}
command in @file{.wgetrc}.  That setting may be overridden from the
command line.  The exception is that, when the output is not a TTY, the
``dot'' progress will be favored over ``bar''.  To force the bar output,
use @samp{--progress=bar:force}.

@item -N
@itemx --timestamping
Turn on time-stamping.  @xref{Time-Stamping}, for details.

@cindex server response, print
@item -S
@itemx --server-response
Print the headers sent by @sc{http} servers and responses sent by
@sc{ftp} servers.

@cindex Wget as spider
@cindex spider
@item --spider
When invoked with this option, Wget will behave as a Web @dfn{spider},
which means that it will not download the pages, just check that they
are there.  You can use it to check your bookmarks, e.g. with:

@example
wget --spider --force-html -i bookmarks.html
@end example

This feature needs much more work for Wget to get close to the
functionality of real @sc{www} spiders.

@cindex timeout
@item -T seconds
@itemx --timeout=@var{seconds}
Set the read timeout to @var{seconds} seconds.  Whenever a network read
is issued, the file descriptor is checked for a timeout, which could
otherwise leave a pending connection (uninterrupted read).  The default
timeout is 900 seconds (fifteen minutes).  Setting timeout to 0 will
disable checking for timeouts.

Please do not lower the default timeout value with this option unless
you know what you are doing.

@cindex bandwidth, limit
@cindex rate, limit
@cindex limit bandwidth
@item --limit-rate=@var{amount}
Limit the download speed to @var{amount} bytes per second.  Amount may
be expressed in bytes, kilobytes with the @samp{k} suffix, or megabytes
with the @samp{m} suffix.  For example, @samp{--limit-rate=20k} will
limit the retrieval rate to 20KB/s.  This kind of thing is useful when,
for whatever reason, you don't want Wget to consume the entire evailable
bandwidth.

Note that Wget implementeds the limiting by sleeping the appropriate
amount of time after a network read that took less time than specified
by the rate.  Eventually this strategy causes the TCP transfer to slow
down to approximately the specified rate.  However, it takes some time
for this balance to be achieved, so don't be surprised if limiting the
rate doesn't work with very small files.  Also, the "sleeping" strategy
will misfire when an extremely small bandwidth, say less than 1.5KB/s,
is specified.

@cindex pause
@cindex wait
@item -w @var{seconds}
@itemx --wait=@var{seconds}
Wait the specified number of seconds between the retrievals.  Use of
this option is recommended, as it lightens the server load by making the
requests less frequent.  Instead of in seconds, the time can be
specified in minutes using the @code{m} suffix, in hours using @code{h}
suffix, or in days using @code{d} suffix.

Specifying a large value for this option is useful if the network or the
destination host is down, so that Wget can wait long enough to
reasonably expect the network error to be fixed before the retry.

@cindex retries, waiting between
@cindex waiting between retries
@item --waitretry=@var{seconds}
If you don't want Wget to wait between @emph{every} retrieval, but only
between retries of failed downloads, you can use this option.  Wget will
use @dfn{linear backoff}, waiting 1 second after the first failure on a
given file, then waiting 2 seconds after the second failure on that
file, up to the maximum number of @var{seconds} you specify.  Therefore,
a value of 10 will actually make Wget wait up to (1 + 2 + ... + 10) = 55
seconds per file.

Note that this option is turned on by default in the global
@file{wgetrc} file.

@cindex wait, random
@cindex random wait
@itemx --random-wait
Some web sites may perform log analysis to identify retrieval programs
such as Wget by looking for statistically significant similarities in
the time between requests. This option causes the time between requests
to vary between 0 and 2 * @var{wait} seconds, where @var{wait} was
specified using the @samp{-w} or @samp{--wait} options, in order to mask
Wget's presence from such analysis.

A recent article in a publication devoted to development on a popular
consumer platform provided code to perform this analysis on the fly.
Its author suggested blocking at the class C address level to ensure
automated retrieval programs were blocked despite changing DHCP-supplied
addresses.

The @samp{--random-wait} option was inspired by this ill-advised
recommendation to block many unrelated users from a web site due to the
actions of one.

@cindex proxy
@item -Y on/off
@itemx --proxy=on/off
Turn proxy support on or off.  The proxy is on by default if the
appropriate environmental variable is defined.

@cindex quota
@item -Q @var{quota}
@itemx --quota=@var{quota}
Specify download quota for automatic retrievals.  The value can be
specified in bytes (default), kilobytes (with @samp{k} suffix), or
megabytes (with @samp{m} suffix).

Note that quota will never affect downloading a single file.  So if you
specify @samp{wget -Q10k ftp://wuarchive.wustl.edu/ls-lR.gz}, all of the
@file{ls-lR.gz} will be downloaded.  The same goes even when several
@sc{url}s are specified on the command-line.  However, quota is
respected when retrieving either recursively, or from an input file.
Thus you may safely type @samp{wget -Q2m -i sites}---download will be
aborted when the quota is exceeded.

Setting quota to 0 or to @samp{inf} unlimits the download quota.
@end table

@node Directory Options, HTTP Options, Download Options, Invoking
@section Directory Options

@table @samp
@item -nd
@itemx --no-directories
Do not create a hierarchy of directories when retrieving recursively.
With this option turned on, all files will get saved to the current
directory, without clobbering (if a name shows up more than once, the
filenames will get extensions @samp{.n}).

@item -x
@itemx --force-directories
The opposite of @samp{-nd}---create a hierarchy of directories, even if
one would not have been created otherwise.  E.g. @samp{wget -x
http://fly.srk.fer.hr/robots.txt} will save the downloaded file to
@file{fly.srk.fer.hr/robots.txt}.

@item -nH
@itemx --no-host-directories
Disable generation of host-prefixed directories.  By default, invoking
Wget with @samp{-r http://fly.srk.fer.hr/} will create a structure of
directories beginning with @file{fly.srk.fer.hr/}.  This option disables
such behavior.

@cindex cut directories
@item --cut-dirs=@var{number}
Ignore @var{number} directory components.  This is useful for getting a
fine-grained control over the directory where recursive retrieval will
be saved.

Take, for example, the directory at
@samp{ftp://ftp.xemacs.org/pub/xemacs/}.  If you retrieve it with
@samp{-r}, it will be saved locally under
@file{ftp.xemacs.org/pub/xemacs/}.  While the @samp{-nH} option can
remove the @file{ftp.xemacs.org/} part, you are still stuck with
@file{pub/xemacs}.  This is where @samp{--cut-dirs} comes in handy; it
makes Wget not ``see'' @var{number} remote directory components.  Here
are several examples of how @samp{--cut-dirs} option works.

@example
@group
No options        -> ftp.xemacs.org/pub/xemacs/
-nH               -> pub/xemacs/
-nH --cut-dirs=1  -> xemacs/
-nH --cut-dirs=2  -> .

--cut-dirs=1      -> ftp.xemacs.org/xemacs/
...
@end group
@end example

If you just want to get rid of the directory structure, this option is
similar to a combination of @samp{-nd} and @samp{-P}.  However, unlike
@samp{-nd}, @samp{--cut-dirs} does not lose with subdirectories---for
instance, with @samp{-nH --cut-dirs=1}, a @file{beta/} subdirectory will
be placed to @file{xemacs/beta}, as one would expect.

@cindex directory prefix
@item -P @var{prefix}
@itemx --directory-prefix=@var{prefix}
Set directory prefix to @var{prefix}.  The @dfn{directory prefix} is the
directory where all other files and subdirectories will be saved to,
i.e. the top of the retrieval tree.  The default is @samp{.} (the
current directory).
@end table

@node HTTP Options, FTP Options, Directory Options, Invoking
@section HTTP Options

@table @samp
@cindex .html extension
@item -E
@itemx --html-extension
If a file of type @samp{text/html} is downloaded and the URL does not
end with the regexp @samp{\.[Hh][Tt][Mm][Ll]?}, this option will cause
the suffix @samp{.html} to be appended to the local filename.  This is
useful, for instance, when you're mirroring a remote site that uses
@samp{.asp} pages, but you want the mirrored pages to be viewable on
your stock Apache server.  Another good use for this is when you're
downloading the output of CGIs.  A URL like
@samp{http://site.com/article.cgi?25} will be saved as
@file{article.cgi?25.html}.

Note that filenames changed in this way will be re-downloaded every time
you re-mirror a site, because Wget can't tell that the local
@file{@var{X}.html} file corresponds to remote URL @samp{@var{X}} (since
it doesn't yet know that the URL produces output of type
@samp{text/html}.  To prevent this re-downloading, you must use
@samp{-k} and @samp{-K} so that the original version of the file will be
saved as @file{@var{X}.orig} (@pxref{Recursive Retrieval Options}).

@cindex http user
@cindex http password
@cindex authentication
@item --http-user=@var{user}
@itemx --http-passwd=@var{password}
Specify the username @var{user} and password @var{password} on an
@sc{http} server.  According to the type of the challenge, Wget will
encode them using either the @code{basic} (insecure) or the
@code{digest} authentication scheme.

Another way to specify username and password is in the @sc{url} itself
(@pxref{URL Format}).  Either method reveals your password to anyone who
bothers to run @code{ps}.  To prevent the passwords from being seen,
store them in @file{.wgetrc} or @file{.netrc}, and make sure to protect
those files from other users with @code{chmod}.  If the passwords are
really important, do not leave them lying in those files either---edit
the files and delete them after Wget has started the download.

For more information about security issues with Wget, @xref{Security
Considerations}.

@cindex proxy
@cindex cache
@item -C on/off
@itemx --cache=on/off
When set to off, disable server-side cache.  In this case, Wget will
send the remote server an appropriate directive (@samp{Pragma:
no-cache}) to get the file from the remote service, rather than
returning the cached version.  This is especially useful for retrieving
and flushing out-of-date documents on proxy servers.

Caching is allowed by default.

@cindex cookies
@item --cookies=on/off
When set to off, disable the use of cookies.  Cookies are a mechanism
for maintaining server-side state.  The server sends the client a cookie
using the @code{Set-Cookie} header, and the client responds with the
same cookie upon further requests.  Since cookies allow the server
owners to keep track of visitors and for sites to exchange this
information, some consider them a breach of privacy.  The default is to
use cookies; however, @emph{storing} cookies is not on by default.

@cindex loading cookies
@cindex cookies, loading
@item --load-cookies @var{file}
Load cookies from @var{file} before the first HTTP retrieval.
@var{file} is a textual file in the format originally used by Netscape's
@file{cookies.txt} file.

You will typically use this option when mirroring sites that require
that you be logged in to access some or all of their content.  The login
process typically works by the web server issuing an @sc{http} cookie
upon receiving and verifying your credentials.  The cookie is then
resent by the browser when accessing that part of the site, and so
proves your identity.

Mirroring such a site requires Wget to send the same cookies your
browser sends when communicating with the site.  This is achieved by
@samp{--load-cookies}---simply point Wget to the location of the
@file{cookies.txt} file, and it will send the same cookies your browser
would send in the same situation.  Different browsers keep textual
cookie files in different locations:

@table @asis
@item Netscape 4.x.
The cookies are in @file{~/.netscape/cookies.txt}.

@item Mozilla and Netscape 6.x.
Mozilla's cookie file is also named @file{cookies.txt}, located
somewhere under @file{~/.mozilla}, in the directory of your profile.
The full path usually ends up looking somewhat like
@file{~/.mozilla/default/@var{some-weird-string}/cookies.txt}.

@item Internet Explorer.
You can produce a cookie file Wget can use by using the File menu,
Import and Export, Export Cookies.  This has been tested with Internet
Explorer 5; it is not guaranteed to work with earlier versions.

@item Other browsers.
If you are using a different browser to create your cookies,
@samp{--load-cookies} will only work if you can locate or produce a
cookie file in the Netscape format that Wget expects.
@end table

If you cannot use @samp{--load-cookies}, there might still be an
alternative.  If your browser supports a ``cookie manager'', you can use
it to view the cookies used when accessing the site you're mirroring.
Write down the name and value of the cookie, and manually instruct Wget
to send those cookies, bypassing the ``official'' cookie support:

@example
wget --cookies=off --header "Cookie: @var{name}=@var{value}"
@end example

@cindex saving cookies
@cindex cookies, saving
@item --save-cookies @var{file}
Save cookies from @var{file} at the end of session.  Cookies whose
expiry time is not specified, or those that have already expired, are
not saved.

@cindex Content-Length, ignore
@cindex ignore length
@item --ignore-length
Unfortunately, some @sc{http} servers (@sc{cgi} programs, to be more
precise) send out bogus @code{Content-Length} headers, which makes Wget
go wild, as it thinks not all the document was retrieved.  You can spot
this syndrome if Wget retries getting the same document again and again,
each time claiming that the (otherwise normal) connection has closed on
the very same byte.

With this option, Wget will ignore the @code{Content-Length} header---as
if it never existed.

@cindex header, add
@item --header=@var{additional-header}
Define an @var{additional-header} to be passed to the @sc{http} servers.
Headers must contain a @samp{:} preceded by one or more non-blank
characters, and must not contain newlines.

You may define more than one additional header by specifying
@samp{--header} more than once.

@example
@group
wget --header='Accept-Charset: iso-8859-2' \
     --header='Accept-Language: hr'        \
       http://fly.srk.fer.hr/
@end group
@end example

Specification of an empty string as the header value will clear all
previous user-defined headers.

@cindex proxy user
@cindex proxy password
@cindex proxy authentication
@item --proxy-user=@var{user}
@itemx --proxy-passwd=@var{password}
Specify the username @var{user} and password @var{password} for
authentication on a proxy server.  Wget will encode them using the
@code{basic} authentication scheme.

Security considerations similar to those with @samp{--http-passwd}
pertain here as well.

@cindex http referer
@cindex referer, http
@item --referer=@var{url}
Include `Referer: @var{url}' header in HTTP request.  Useful for
retrieving documents with server-side processing that assume they are
always being retrieved by interactive web browsers and only come out
properly when Referer is set to one of the pages that point to them.

@cindex server response, save
@item -s
@itemx --save-headers
Save the headers sent by the @sc{http} server to the file, preceding the
actual contents, with an empty line as the separator.

@cindex user-agent
@item -U @var{agent-string}
@itemx --user-agent=@var{agent-string}
Identify as @var{agent-string} to the @sc{http} server.

The @sc{http} protocol allows the clients to identify themselves using a
@code{User-Agent} header field.  This enables distinguishing the
@sc{www} software, usually for statistical purposes or for tracing of
protocol violations.  Wget normally identifies as
@samp{Wget/@var{version}}, @var{version} being the current version
number of Wget.

However, some sites have been known to impose the policy of tailoring
the output according to the @code{User-Agent}-supplied information.
While conceptually this is not such a bad idea, it has been abused by
servers denying information to clients other than @code{Mozilla} or
Microsoft @code{Internet Explorer}.  This option allows you to change
the @code{User-Agent} line issued by Wget.  Use of this option is
discouraged, unless you really know what you are doing.
@end table

@node FTP Options, Recursive Retrieval Options, HTTP Options, Invoking
@section FTP Options

@table @samp
@cindex .listing files, removing
@item -nr
@itemx --dont-remove-listing
Don't remove the temporary @file{.listing} files generated by @sc{ftp}
retrievals.  Normally, these files contain the raw directory listings
received from @sc{ftp} servers.  Not removing them can be useful for
debugging purposes, or when you want to be able to easily check on the
contents of remote server directories (e.g. to verify that a mirror
you're running is complete).

Note that even though Wget writes to a known filename for this file,
this is not a security hole in the scenario of a user making
@file{.listing} a symbolic link to @file{/etc/passwd} or something and
asking @code{root} to run Wget in his or her directory.  Depending on
the options used, either Wget will refuse to write to @file{.listing},
making the globbing/recursion/time-stamping operation fail, or the
symbolic link will be deleted and replaced with the actual
@file{.listing} file, or the listing will be written to a
@file{.listing.@var{number}} file.

Even though this situation isn't a problem, though, @code{root} should
never run Wget in a non-trusted user's directory.  A user could do
something as simple as linking @file{index.html} to @file{/etc/passwd}
and asking @code{root} to run Wget with @samp{-N} or @samp{-r} so the file
will be overwritten.

@cindex globbing, toggle
@item -g on/off
@itemx --glob=on/off
Turn @sc{ftp} globbing on or off.  Globbing means you may use the
shell-like special characters (@dfn{wildcards}), like @samp{*},
@samp{?}, @samp{[} and @samp{]} to retrieve more than one file from the
same directory at once, like:

@example
wget ftp://gnjilux.srk.fer.hr/*.msg
@end example

By default, globbing will be turned on if the @sc{url} contains a
globbing character.  This option may be used to turn globbing on or off
permanently.

You may have to quote the @sc{url} to protect it from being expanded by
your shell.  Globbing makes Wget look for a directory listing, which is
system-specific.  This is why it currently works only with Unix @sc{ftp}
servers (and the ones emulating Unix @code{ls} output).

@cindex passive ftp
@item --passive-ftp
Use the @dfn{passive} @sc{ftp} retrieval scheme, in which the client
initiates the data connection.  This is sometimes required for @sc{ftp}
to work behind firewalls.

@cindex symbolic links, retrieving
@item --retr-symlinks
Usually, when retrieving @sc{ftp} directories recursively and a symbolic
link is encountered, the linked-to file is not downloaded.  Instead, a
matching symbolic link is created on the local filesystem.  The
pointed-to file will not be downloaded unless this recursive retrieval
would have encountered it separately and downloaded it anyway.

When @samp{--retr-symlinks} is specified, however, symbolic links are
traversed and the pointed-to files are retrieved.  At this time, this
option does not cause Wget to traverse symlinks to directories and
recurse through them, but in the future it should be enhanced to do
this.

Note that when retrieving a file (not a directory) because it was
specified on the commandline, rather than because it was recursed to,
this option has no effect.  Symbolic links are always traversed in this
case.
@end table

@node Recursive Retrieval Options, Recursive Accept/Reject Options, FTP Options, Invoking
@section Recursive Retrieval Options

@table @samp
@item -r
@itemx --recursive
Turn on recursive retrieving.  @xref{Recursive Retrieval}, for more
details.

@item -l @var{depth}
@itemx --level=@var{depth}
Specify recursion maximum depth level @var{depth} (@pxref{Recursive
Retrieval}).  The default maximum depth is 5.

@cindex proxy filling
@cindex delete after retrieval
@cindex filling proxy cache
@item --delete-after
This option tells Wget to delete every single file it downloads,
@emph{after} having done so.  It is useful for pre-fetching popular
pages through a proxy, e.g.:

@example
wget -r -nd --delete-after http://whatever.com/~popular/page/
@end example

The @samp{-r} option is to retrieve recursively, and @samp{-nd} to not
create directories.  

Note that @samp{--delete-after} deletes files on the local machine.  It
does not issue the @samp{DELE} command to remote FTP sites, for
instance.  Also note that when @samp{--delete-after} is specified,
@samp{--convert-links} is ignored, so @samp{.orig} files are simply not
created in the first place.

@cindex conversion of links
@cindex link conversion
@item -k
@itemx --convert-links
After the download is complete, convert the links in the document to
make them suitable for local viewing.  This affects not only the visible
hyperlinks, but any part of the document that links to external content,
such as embedded images, links to style sheets, hyperlinks to non-HTML
content, etc.

Each link will be changed in one of the two ways:

@itemize @bullet
@item
The links to files that have been downloaded by Wget will be changed to
refer to the file they point to as a relative link.

Example: if the downloaded file @file{/foo/doc.html} links to
@file{/bar/img.gif}, also downloaded, then the link in @file{doc.html}
will be modified to point to @samp{../bar/img.gif}.  This kind of
transformation works reliably for arbitrary combinations of directories.

@item
The links to files that have not been downloaded by Wget will be changed
to include host name and absolute path of the location they point to.

Example: if the downloaded file @file{/foo/doc.html} links to
@file{/bar/img.gif} (or to @file{../bar/img.gif}), then the link in
@file{doc.html} will be modified to point to
@file{http://@var{hostname}/bar/img.gif}.
@end itemize

Because of this, local browsing works reliably: if a linked file was
downloaded, the link will refer to its local name; if it was not
downloaded, the link will refer to its full Internet address rather than
presenting a broken link.  The fact that the former links are converted
to relative links ensures that you can move the downloaded hierarchy to
another directory.

Note that only at the end of the download can Wget know which links have
been downloaded.  Because of that, the work done by @samp{-k} will be
performed at the end of all the downloads.

@cindex backing up converted files
@item -K
@itemx --backup-converted
When converting a file, back up the original version with a @samp{.orig}
suffix.  Affects the behavior of @samp{-N} (@pxref{HTTP Time-Stamping
Internals}).

@item -m
@itemx --mirror
Turn on options suitable for mirroring.  This option turns on recursion
and time-stamping, sets infinite recursion depth and keeps @sc{ftp}
directory listings.  It is currently equivalent to
@samp{-r -N -l inf -nr}.

@cindex page requisites
@cindex required images, downloading
@item -p
@itemx --page-requisites
This option causes Wget to download all the files that are necessary to
properly display a given HTML page.  This includes such things as
inlined images, sounds, and referenced stylesheets.

Ordinarily, when downloading a single HTML page, any requisite documents
that may be needed to display it properly are not downloaded.  Using
@samp{-r} together with @samp{-l} can help, but since Wget does not
ordinarily distinguish between external and inlined documents, one is
generally left with ``leaf documents'' that are missing their
requisites.

For instance, say document @file{1.html} contains an @code{<IMG>} tag
referencing @file{1.gif} and an @code{<A>} tag pointing to external
document @file{2.html}.  Say that @file{2.html} is similar but that its
image is @file{2.gif} and it links to @file{3.html}.  Say this
continues up to some arbitrarily high number.

If one executes the command:

@example
wget -r -l 2 http://@var{site}/1.html
@end example

then @file{1.html}, @file{1.gif}, @file{2.html}, @file{2.gif}, and
@file{3.html} will be downloaded.  As you can see, @file{3.html} is
without its requisite @file{3.gif} because Wget is simply counting the
number of hops (up to 2) away from @file{1.html} in order to determine
where to stop the recursion.  However, with this command:

@example
wget -r -l 2 -p http://@var{site}/1.html
@end example

all the above files @emph{and} @file{3.html}'s requisite @file{3.gif}
will be downloaded.  Similarly,

@example
wget -r -l 1 -p http://@var{site}/1.html
@end example

will cause @file{1.html}, @file{1.gif}, @file{2.html}, and @file{2.gif}
to be downloaded.  One might think that:

@example
wget -r -l 0 -p http://@var{site}/1.html
@end example

would download just @file{1.html} and @file{1.gif}, but unfortunately
this is not the case, because @samp{-l 0} is equivalent to
@samp{-l inf}---that is, infinite recursion.  To download a single HTML
page (or a handful of them, all specified on the commandline or in a
@samp{-i} @sc{url} input file) and its (or their) requisites, simply leave off
@samp{-r} and @samp{-l}:

@example
wget -p http://@var{site}/1.html
@end example

Note that Wget will behave as if @samp{-r} had been specified, but only
that single page and its requisites will be downloaded.  Links from that
page to external documents will not be followed.  Actually, to download
a single page and all its requisites (even if they exist on separate
websites), and make sure the lot displays properly locally, this author
likes to use a few options in addition to @samp{-p}:

@example
wget -E -H -k -K -p http://@var{site}/@var{document}
@end example

To finish off this topic, it's worth knowing that Wget's idea of an
external document link is any URL specified in an @code{<A>} tag, an
@code{<AREA>} tag, or a @code{<LINK>} tag other than @code{<LINK
REL="stylesheet">}.
@end table

@node Recursive Accept/Reject Options,  , Recursive Retrieval Options, Invoking
@section Recursive Accept/Reject Options

@table @samp
@item -A @var{acclist} --accept @var{acclist}
@itemx -R @var{rejlist} --reject @var{rejlist}
Specify comma-separated lists of file name suffixes or patterns to
accept or reject (@pxref{Types of Files} for more details).

@item -D @var{domain-list}
@itemx --domains=@var{domain-list}
Set domains to be followed.  @var{domain-list} is a comma-separated list
of domains.  Note that it does @emph{not} turn on @samp{-H}.

@item --exclude-domains @var{domain-list}
Specify the domains that are @emph{not} to be followed.
(@pxref{Spanning Hosts}).

@cindex follow FTP links
@item --follow-ftp
Follow @sc{ftp} links from @sc{html} documents.  Without this option,
Wget will ignore all the @sc{ftp} links.

@cindex tag-based recursive pruning
@item --follow-tags=@var{list}
Wget has an internal table of HTML tag / attribute pairs that it
considers when looking for linked documents during a recursive
retrieval.  If a user wants only a subset of those tags to be
considered, however, he or she should be specify such tags in a
comma-separated @var{list} with this option.

@item -G @var{list}
@itemx --ignore-tags=@var{list}
This is the opposite of the @samp{--follow-tags} option.  To skip
certain HTML tags when recursively looking for documents to download,
specify them in a comma-separated @var{list}.  

In the past, the @samp{-G} option was the best bet for downloading a
single page and its requisites, using a commandline like:

@example
wget -Ga,area -H -k -K -r http://@var{site}/@var{document}
@end example

However, the author of this option came across a page with tags like
@code{<LINK REL="home" HREF="/">} and came to the realization that
@samp{-G} was not enough.  One can't just tell Wget to ignore
@code{<LINK>}, because then stylesheets will not be downloaded.  Now the
best bet for downloading a single page and its requisites is the
dedicated @samp{--page-requisites} option.

@item -H
@itemx --span-hosts
Enable spanning across hosts when doing recursive retrieving
(@pxref{Spanning Hosts}).

@item -L
@itemx --relative
Follow relative links only.  Useful for retrieving a specific home page
without any distractions, not even those from the same hosts
(@pxref{Relative Links}).

@item -I @var{list}
@itemx --include-directories=@var{list}
Specify a comma-separated list of directories you wish to follow when
downloading (@pxref{Directory-Based Limits} for more details.)  Elements
of @var{list} may contain wildcards.

@item -X @var{list}
@itemx --exclude-directories=@var{list}
Specify a comma-separated list of directories you wish to exclude from
download (@pxref{Directory-Based Limits} for more details.)  Elements of
@var{list} may contain wildcards.

@item -np
@item --no-parent
Do not ever ascend to the parent directory when retrieving recursively.
This is a useful option, since it guarantees that only the files
@emph{below} a certain hierarchy will be downloaded.
@xref{Directory-Based Limits}, for more details.
@end table

@c man end

@node Recursive Retrieval, Following Links, Invoking, Top
@chapter Recursive Retrieval
@cindex recursion
@cindex retrieving
@cindex recursive retrieval

GNU Wget is capable of traversing parts of the Web (or a single
@sc{http} or @sc{ftp} server), following links and directory structure.
We refer to this as to @dfn{recursive retrieving}, or @dfn{recursion}.

With @sc{http} @sc{url}s, Wget retrieves and parses the @sc{html} from
the given @sc{url}, documents, retrieving the files the @sc{html}
document was referring to, through markups like @code{href}, or
@code{src}.  If the freshly downloaded file is also of type
@code{text/html}, it will be parsed and followed further.

Recursive retrieval of @sc{http} and @sc{html} content is
@dfn{breadth-first}.  This means that Wget first downloads the requested
HTML document, then the documents linked from that document, then the
documents linked by them, and so on.  In other words, Wget first
downloads the documents at depth 1, then those at depth 2, and so on
until the specified maximum depth.

The maximum @dfn{depth} to which the retrieval may descend is specified
with the @samp{-l} option.  The default maximum depth is five layers.

When retrieving an @sc{ftp} @sc{url} recursively, Wget will retrieve all
the data from the given directory tree (including the subdirectories up
to the specified depth) on the remote server, creating its mirror image
locally.  @sc{ftp} retrieval is also limited by the @code{depth}
parameter.  Unlike @sc{http} recursion, @sc{ftp} recursion is performed
depth-first.

By default, Wget will create a local directory tree, corresponding to
the one found on the remote server.

Recursive retrieving can find a number of applications, the most
important of which is mirroring.  It is also useful for @sc{www}
presentations, and any other opportunities where slow network
connections should be bypassed by storing the files locally.

You should be warned that recursive downloads can overload the remote
servers.  Because of that, many administrators frown upon them and may
ban access from your site if they detect very fast downloads of big
amounts of content.  When downloading from Internet servers, consider
using the @samp{-w} option to introduce a delay between accesses to the
server.  The download will take a while longer, but the server
administrator will not be alarmed by your rudeness.

Of course, recursive download may cause problems on your machine.  If
left to run unchecked, it can easily fill up the disk.  If downloading
from local network, it can also take bandwidth on the system, as well as
consume memory and CPU.

Try to specify the criteria that match the kind of download you are
trying to achieve.  If you want to download only one page, use
@samp{--page-requisites} without any additional recursion.  If you want
to download things under one directory, use @samp{-np} to avoid
downloading things from other directories.  If you want to download all
the files from one directory, use @samp{-l 1} to make sure the recursion
depth never exceeds one.  @xref{Following Links}, for more information
about this.

Recursive retrieval should be used with care.  Don't say you were not
warned.

@node Following Links, Time-Stamping, Recursive Retrieval, Top
@chapter Following Links
@cindex links
@cindex following links

When retrieving recursively, one does not wish to retrieve loads of
unnecessary data.  Most of the time the users bear in mind exactly what
they want to download, and want Wget to follow only specific links.

For example, if you wish to download the music archive from
@samp{fly.srk.fer.hr}, you will not want to download all the home pages
that happen to be referenced by an obscure part of the archive.

Wget possesses several mechanisms that allows you to fine-tune which
links it will follow.

@menu
* Spanning Hosts::         (Un)limiting retrieval based on host name.
* Types of Files::         Getting only certain files.
* Directory-Based Limits:: Getting only certain directories.
* Relative Links::         Follow relative links only.
* FTP Links::              Following FTP links.
@end menu

@node Spanning Hosts, Types of Files, Following Links, Following Links
@section Spanning Hosts
@cindex spanning hosts
@cindex hosts, spanning

Wget's recursive retrieval normally refuses to visit hosts different
than the one you specified on the command line.  This is a reasonable
default; without it, every retrieval would have the potential to turn
your Wget into a small version of google.

However, visiting different hosts, or @dfn{host spanning,} is sometimes
a useful option.  Maybe the images are served from a different server.
Maybe you're mirroring a site that consists of pages interlinked between
three servers.  Maybe the server has two equivalent names, and the HTML
pages refer to both interchangeably.

@table @asis
@item Span to any host---@samp{-H}

The @samp{-H} option turns on host spanning, thus allowing Wget's
recursive run to visit any host referenced by a link.  Unless sufficient
recursion-limiting criteria are applied depth, these foreign hosts will
typically link to yet more hosts, and so on until Wget ends up sucking
up much more data than you have intended.

@item Limit spanning to certain domains---@samp{-D}

The @samp{-D} option allows you to specify the domains that will be
followed, thus limiting the recursion only to the hosts that belong to
these domains.  Obviously, this makes sense only in conjunction with
@samp{-H}.  A typical example would be downloading the contents of
@samp{www.server.com}, but allowing downloads from
@samp{images.server.com}, etc.:

@example
wget -rH -Dserver.com http://www.server.com/
@end example

You can specify more than one address by separating them with a comma,
e.g. @samp{-Ddomain1.com,domain2.com}.

@item Keep download off certain domains---@samp{--exclude-domains}

If there are domains you want to exclude specifically, you can do it
with @samp{--exclude-domains}, which accepts the same type of arguments
of @samp{-D}, but will @emph{exclude} all the listed domains.  For
example, if you want to download all the hosts from @samp{foo.edu}
domain, with the exception of @samp{sunsite.foo.edu}, you can do it like
this:

@example
wget -rH -Dfoo.edu --exclude-domains sunsite.foo.edu \
    http://www.foo.edu/
@end example

@end table

@node Types of Files, Directory-Based Limits, Spanning Hosts, Following Links
@section Types of Files
@cindex types of files

When downloading material from the web, you will often want to restrict
the retrieval to only certain file types.  For example, if you are
interested in downloading @sc{gif}s, you will not be overjoyed to get
loads of PostScript documents, and vice versa.

Wget offers two options to deal with this problem.  Each option
description lists a short name, a long name, and the equivalent command
in @file{.wgetrc}.

@cindex accept wildcards
@cindex accept suffixes
@cindex wildcards, accept
@cindex suffixes, accept
@table @samp
@item -A @var{acclist}
@itemx --accept @var{acclist}
@itemx accept = @var{acclist}
The argument to @samp{--accept} option is a list of file suffixes or
patterns that Wget will download during recursive retrieval.  A suffix
is the ending part of a file, and consists of ``normal'' letters,
e.g. @samp{gif} or @samp{.jpg}.  A matching pattern contains shell-like
wildcards, e.g. @samp{books*} or @samp{zelazny*196[0-9]*}.

So, specifying @samp{wget -A gif,jpg} will make Wget download only the
files ending with @samp{gif} or @samp{jpg}, i.e. @sc{gif}s and
@sc{jpeg}s.  On the other hand, @samp{wget -A "zelazny*196[0-9]*"} will
download only files beginning with @samp{zelazny} and containing numbers
from 1960 to 1969 anywhere within.  Look up the manual of your shell for
a description of how pattern matching works.

Of course, any number of suffixes and patterns can be combined into a
comma-separated list, and given as an argument to @samp{-A}.

@cindex reject wildcards
@cindex reject suffixes
@cindex wildcards, reject
@cindex suffixes, reject
@item -R @var{rejlist}
@itemx --reject @var{rejlist}
@itemx reject = @var{rejlist}
The @samp{--reject} option works the same way as @samp{--accept}, only
its logic is the reverse; Wget will download all files @emph{except} the
ones matching the suffixes (or patterns) in the list.

So, if you want to download a whole page except for the cumbersome
@sc{mpeg}s and @sc{.au} files, you can use @samp{wget -R mpg,mpeg,au}.
Analogously, to download all files except the ones beginning with
@samp{bjork}, use @samp{wget -R "bjork*"}.  The quotes are to prevent
expansion by the shell.
@end table

The @samp{-A} and @samp{-R} options may be combined to achieve even
better fine-tuning of which files to retrieve.  E.g. @samp{wget -A
"*zelazny*" -R .ps} will download all the files having @samp{zelazny} as
a part of their name, but @emph{not} the PostScript files.

Note that these two options do not affect the downloading of @sc{html}
files; Wget must load all the @sc{html}s to know where to go at
all---recursive retrieval would make no sense otherwise.

@node Directory-Based Limits, Relative Links, Types of Files, Following Links
@section Directory-Based Limits
@cindex directories
@cindex directory limits

Regardless of other link-following facilities, it is often useful to
place the restriction of what files to retrieve based on the directories
those files are placed in.  There can be many reasons for this---the
home pages may be organized in a reasonable directory structure; or some
directories may contain useless information, e.g. @file{/cgi-bin} or
@file{/dev} directories.

Wget offers three different options to deal with this requirement.  Each
option description lists a short name, a long name, and the equivalent
command in @file{.wgetrc}.

@cindex directories, include
@cindex include directories
@cindex accept directories
@table @samp
@item -I @var{list}
@itemx --include @var{list}
@itemx include_directories = @var{list}
@samp{-I} option accepts a comma-separated list of directories included
in the retrieval.  Any other directories will simply be ignored.  The
directories are absolute paths.

So, if you wish to download from @samp{http://host/people/bozo/}
following only links to bozo's colleagues in the @file{/people}
directory and the bogus scripts in @file{/cgi-bin}, you can specify:

@example
wget -I /people,/cgi-bin http://host/people/bozo/
@end example

@cindex directories, exclude
@cindex exclude directories
@cindex reject directories
@item -X @var{list}
@itemx --exclude @var{list}
@itemx exclude_directories = @var{list}
@samp{-X} option is exactly the reverse of @samp{-I}---this is a list of
directories @emph{excluded} from the download.  E.g. if you do not want
Wget to download things from @file{/cgi-bin} directory, specify @samp{-X
/cgi-bin} on the command line.

The same as with @samp{-A}/@samp{-R}, these two options can be combined
to get a better fine-tuning of downloading subdirectories.  E.g. if you
want to load all the files from @file{/pub} hierarchy except for
@file{/pub/worthless}, specify @samp{-I/pub -X/pub/worthless}.

@cindex no parent
@item -np
@itemx --no-parent
@itemx no_parent = on
The simplest, and often very useful way of limiting directories is
disallowing retrieval of the links that refer to the hierarchy
@dfn{above} than the beginning directory, i.e. disallowing ascent to the
parent directory/directories.

The @samp{--no-parent} option (short @samp{-np}) is useful in this case.
Using it guarantees that you will never leave the existing hierarchy.
Supposing you issue Wget with:

@example
wget -r --no-parent http://somehost/~luzer/my-archive/
@end example

You may rest assured that none of the references to
@file{/~his-girls-homepage/} or @file{/~luzer/all-my-mpegs/} will be
followed.  Only the archive you are interested in will be downloaded.
Essentially, @samp{--no-parent} is similar to
@samp{-I/~luzer/my-archive}, only it handles redirections in a more
intelligent fashion.
@end table

@node Relative Links, FTP Links, Directory-Based Limits, Following Links
@section Relative Links
@cindex relative links

When @samp{-L} is turned on, only the relative links are ever followed.
Relative links are here defined those that do not refer to the web
server root.  For example, these links are relative:

@example
<a href="foo.gif">
<a href="foo/bar.gif">
<a href="../foo/bar.gif">
@end example

These links are not relative:

@example
<a href="/foo.gif">
<a href="/foo/bar.gif">
<a href="http://www.server.com/foo/bar.gif">
@end example

Using this option guarantees that recursive retrieval will not span
hosts, even without @samp{-H}.  In simple cases it also allows downloads
to ``just work'' without having to convert links.

This option is probably not very useful and might be removed in a future
release.

@node FTP Links,  , Relative Links, Following Links
@section Following FTP Links
@cindex following ftp links

The rules for @sc{ftp} are somewhat specific, as it is necessary for
them to be.  @sc{ftp} links in @sc{html} documents are often included
for purposes of reference, and it is often inconvenient to download them
by default.

To have @sc{ftp} links followed from @sc{html} documents, you need to
specify the @samp{--follow-ftp} option.  Having done that, @sc{ftp}
links will span hosts regardless of @samp{-H} setting.  This is logical,
as @sc{ftp} links rarely point to the same host where the @sc{http}
server resides.  For similar reasons, the @samp{-L} options has no
effect on such downloads.  On the other hand, domain acceptance
(@samp{-D}) and suffix rules (@samp{-A} and @samp{-R}) apply normally.

Also note that followed links to @sc{ftp} directories will not be
retrieved recursively further.

@node Time-Stamping, Startup File, Following Links, Top
@chapter Time-Stamping
@cindex time-stamping
@cindex timestamping
@cindex updating the archives
@cindex incremental updating

One of the most important aspects of mirroring information from the
Internet is updating your archives.

Downloading the whole archive again and again, just to replace a few
changed files is expensive, both in terms of wasted bandwidth and money,
and the time to do the update.  This is why all the mirroring tools
offer the option of incremental updating.

Such an updating mechanism means that the remote server is scanned in
search of @dfn{new} files.  Only those new files will be downloaded in
the place of the old ones.

A file is considered new if one of these two conditions are met:

@enumerate
@item
A file of that name does not already exist locally.

@item
A file of that name does exist, but the remote file was modified more
recently than the local file.
@end enumerate

To implement this, the program needs to be aware of the time of last
modification of both local and remote files.  We call this information the
@dfn{time-stamp} of a file.

The time-stamping in GNU Wget is turned on using @samp{--timestamping}
(@samp{-N}) option, or through @code{timestamping = on} directive in
@file{.wgetrc}.  With this option, for each file it intends to download,
Wget will check whether a local file of the same name exists.  If it
does, and the remote file is older, Wget will not download it.

If the local file does not exist, or the sizes of the files do not
match, Wget will download the remote file no matter what the time-stamps
say.

@menu
* Time-Stamping Usage::
* HTTP Time-Stamping Internals::
* FTP Time-Stamping Internals::
@end menu

@node Time-Stamping Usage, HTTP Time-Stamping Internals, Time-Stamping, Time-Stamping
@section Time-Stamping Usage
@cindex time-stamping usage
@cindex usage, time-stamping

The usage of time-stamping is simple.  Say you would like to download a
file so that it keeps its date of modification.

@example
wget -S http://www.gnu.ai.mit.edu/
@end example

A simple @code{ls -l} shows that the time stamp on the local file equals
the state of the @code{Last-Modified} header, as returned by the server.
As you can see, the time-stamping info is preserved locally, even
without @samp{-N} (at least for @sc{http}).

Several days later, you would like Wget to check if the remote file has
changed, and download it if it has.

@example
wget -N http://www.gnu.ai.mit.edu/
@end example

Wget will ask the server for the last-modified date.  If the local file
has the same timestamp as the server, or a newer one, the remote file
will not be re-fetched.  However, if the remote file is more recent,
Wget will proceed to fetch it.

The same goes for @sc{ftp}.  For example:

@example
wget "ftp://ftp.ifi.uio.no/pub/emacs/gnus/*"
@end example

(The quotes around that URL are to prevent the shell from trying to
interpret the @samp{*}.)

After download, a local directory listing will show that the timestamps
match those on the remote server.  Reissuing the command with @samp{-N}
will make Wget re-fetch @emph{only} the files that have been modified
since the last download.

If you wished to mirror the GNU archive every week, you would use a
command like the following, weekly:

@example
wget --timestamping -r ftp://ftp.gnu.org/pub/gnu/
@end example

Note that time-stamping will only work for files for which the server
gives a timestamp.  For @sc{http}, this depends on getting a
@code{Last-Modified} header.  For @sc{ftp}, this depends on getting a
directory listing with dates in a format that Wget can parse
(@pxref{FTP Time-Stamping Internals}).

@node HTTP Time-Stamping Internals, FTP Time-Stamping Internals, Time-Stamping Usage, Time-Stamping
@section HTTP Time-Stamping Internals
@cindex http time-stamping

Time-stamping in @sc{http} is implemented by checking of the
@code{Last-Modified} header.  If you wish to retrieve the file
@file{foo.html} through @sc{http}, Wget will check whether
@file{foo.html} exists locally.  If it doesn't, @file{foo.html} will be
retrieved unconditionally.

If the file does exist locally, Wget will first check its local
time-stamp (similar to the way @code{ls -l} checks it), and then send a
@code{HEAD} request to the remote server, demanding the information on
the remote file.

The @code{Last-Modified} header is examined to find which file was
modified more recently (which makes it ``newer'').  If the remote file
is newer, it will be downloaded; if it is older, Wget will give
up.@footnote{As an additional check, Wget will look at the
@code{Content-Length} header, and compare the sizes; if they are not the
same, the remote file will be downloaded no matter what the time-stamp
says.}

When @samp{--backup-converted} (@samp{-K}) is specified in conjunction
with @samp{-N}, server file @samp{@var{X}} is compared to local file
@samp{@var{X}.orig}, if extant, rather than being compared to local file
@samp{@var{X}}, which will always differ if it's been converted by
@samp{--convert-links} (@samp{-k}).

Arguably, @sc{http} time-stamping should be implemented using the
@code{If-Modified-Since} request.

@node FTP Time-Stamping Internals,  , HTTP Time-Stamping Internals, Time-Stamping
@section FTP Time-Stamping Internals
@cindex ftp time-stamping

In theory, @sc{ftp} time-stamping works much the same as @sc{http}, only
@sc{ftp} has no headers---time-stamps must be ferreted out of directory
listings.

If an @sc{ftp} download is recursive or uses globbing, Wget will use the
@sc{ftp} @code{LIST} command to get a file listing for the directory
containing the desired file(s).  It will try to analyze the listing,
treating it like Unix @code{ls -l} output, extracting the time-stamps.
The rest is exactly the same as for @sc{http}.  Note that when
retrieving individual files from an @sc{ftp} server without using
globbing or recursion, listing files will not be downloaded (and thus
files will not be time-stamped) unless @samp{-N} is specified.

Assumption that every directory listing is a Unix-style listing may
sound extremely constraining, but in practice it is not, as many
non-Unix @sc{ftp} servers use the Unixoid listing format because most
(all?) of the clients understand it.  Bear in mind that @sc{rfc959}
defines no standard way to get a file list, let alone the time-stamps.
We can only hope that a future standard will define this.

Another non-standard solution includes the use of @code{MDTM} command
that is supported by some @sc{ftp} servers (including the popular
@code{wu-ftpd}), which returns the exact time of the specified file.
Wget may support this command in the future.

@node Startup File, Examples, Time-Stamping, Top
@chapter Startup File
@cindex startup file
@cindex wgetrc
@cindex .wgetrc
@cindex startup
@cindex .netrc

Once you know how to change default settings of Wget through command
line arguments, you may wish to make some of those settings permanent.
You can do that in a convenient way by creating the Wget startup
file---@file{.wgetrc}.

Besides @file{.wgetrc} is the ``main'' initialization file, it is
convenient to have a special facility for storing passwords.  Thus Wget
reads and interprets the contents of @file{$HOME/.netrc}, if it finds
it.  You can find @file{.netrc} format in your system manuals.

Wget reads @file{.wgetrc} upon startup, recognizing a limited set of
commands.

@menu
* Wgetrc Location::   Location of various wgetrc files.
* Wgetrc Syntax::     Syntax of wgetrc.
* Wgetrc Commands::   List of available commands.
* Sample Wgetrc::     A wgetrc example.
@end menu

@node Wgetrc Location, Wgetrc Syntax, Startup File, Startup File
@section Wgetrc Location
@cindex wgetrc location
@cindex location of wgetrc

When initializing, Wget will look for a @dfn{global} startup file,
@file{/usr/local/etc/wgetrc} by default (or some prefix other than
@file{/usr/local}, if Wget was not installed there) and read commands
from there, if it exists.

Then it will look for the user's file.  If the environmental variable
@code{WGETRC} is set, Wget will try to load that file.  Failing that, no
further attempts will be made.

If @code{WGETRC} is not set, Wget will try to load @file{$HOME/.wgetrc}.

The fact that user's settings are loaded after the system-wide ones
means that in case of collision user's wgetrc @emph{overrides} the
system-wide wgetrc (in @file{/usr/local/etc/wgetrc} by default).
Fascist admins, away!

@node Wgetrc Syntax, Wgetrc Commands, Wgetrc Location, Startup File
@section Wgetrc Syntax
@cindex wgetrc syntax
@cindex syntax of wgetrc

The syntax of a wgetrc command is simple:

@example
variable = value
@end example

The @dfn{variable} will also be called @dfn{command}.  Valid
@dfn{values} are different for different commands.

The commands are case-insensitive and underscore-insensitive.  Thus
@samp{DIr__PrefiX} is the same as @samp{dirprefix}.  Empty lines, lines
beginning with @samp{#} and lines containing white-space only are
discarded.

Commands that expect a comma-separated list will clear the list on an
empty command.  So, if you wish to reset the rejection list specified in
global @file{wgetrc}, you can do it with:

@example
reject =
@end example

@node Wgetrc Commands, Sample Wgetrc, Wgetrc Syntax, Startup File
@section Wgetrc Commands
@cindex wgetrc commands

The complete set of commands is listed below.  Legal values are listed
after the @samp{=}.  Simple Boolean values can be set or unset using
@samp{on} and @samp{off} or @samp{1} and @samp{0}.  A fancier kind of
Boolean allowed in some cases is the @dfn{lockable Boolean}, which may
be set to @samp{on}, @samp{off}, @samp{always}, or @samp{never}.  If an
option is set to @samp{always} or @samp{never}, that value will be
locked in for the duration of the Wget invocation---commandline options
will not override.

Some commands take pseudo-arbitrary values.  @var{address} values can be
hostnames or dotted-quad IP addresses.  @var{n} can be any positive
integer, or @samp{inf} for infinity, where appropriate.  @var{string}
values can be any non-empty string.

Most of these commands have commandline equivalents (@pxref{Invoking}),
though some of the more obscure or rarely used ones do not.

@table @asis
@item accept/reject = @var{string}
Same as @samp{-A}/@samp{-R} (@pxref{Types of Files}).

@item add_hostdir = on/off
Enable/disable host-prefixed file names.  @samp{-nH} disables it.

@item continue = on/off
If set to on, force continuation of preexistent partially retrieved
files.  See @samp{-c} before setting it.

@item background = on/off
Enable/disable going to background---the same as @samp{-b} (which
enables it).

@item backup_converted = on/off
Enable/disable saving pre-converted files with the suffix
@samp{.orig}---the same as @samp{-K} (which enables it).

@c @item backups = @var{number}
@c #### Document me!
@c
@item base = @var{string}
Consider relative @sc{url}s in @sc{url} input files forced to be
interpreted as @sc{html} as being relative to @var{string}---the same as
@samp{-B}.

@item bind_address = @var{address}
Bind to @var{address}, like the @samp{--bind-address} option.

@item cache = on/off
When set to off, disallow server-caching.  See the @samp{-C} option.

@item convert links = on/off
Convert non-relative links locally.  The same as @samp{-k}.

@item cookies = on/off
When set to off, disallow cookies.  See the @samp{--cookies} option.

@item load_cookies = @var{file}
Load cookies from @var{file}.  See @samp{--load-cookies}.

@item save_cookies = @var{file}
Save cookies to @var{file}.  See @samp{--save-cookies}.

@item cut_dirs = @var{n}
Ignore @var{n} remote directory components.

@item debug = on/off
Debug mode, same as @samp{-d}.

@item delete_after = on/off
Delete after download---the same as @samp{--delete-after}.

@item dir_prefix = @var{string}
Top of directory tree---the same as @samp{-P}.

@item dirstruct = on/off
Turning dirstruct on or off---the same as @samp{-x} or @samp{-nd},
respectively.

@item domains = @var{string}
Same as @samp{-D} (@pxref{Spanning Hosts}).

@item dot_bytes = @var{n}
Specify the number of bytes ``contained'' in a dot, as seen throughout
the retrieval (1024 by default).  You can postfix the value with
@samp{k} or @samp{m}, representing kilobytes and megabytes,
respectively.  With dot settings you can tailor the dot retrieval to
suit your needs, or you can use the predefined @dfn{styles}
(@pxref{Download Options}).

@item dots_in_line = @var{n}
Specify the number of dots that will be printed in each line throughout
the retrieval (50 by default).

@item dot_spacing = @var{n}
Specify the number of dots in a single cluster (10 by default).

@item exclude_directories = @var{string}
Specify a comma-separated list of directories you wish to exclude from
download---the same as @samp{-X} (@pxref{Directory-Based Limits}).

@item exclude_domains = @var{string}
Same as @samp{--exclude-domains} (@pxref{Spanning Hosts}).

@item follow_ftp = on/off
Follow @sc{ftp} links from @sc{html} documents---the same as
@samp{--follow-ftp}.

@item follow_tags = @var{string}
Only follow certain HTML tags when doing a recursive retrieval, just like
@samp{--follow-tags}.

@item force_html = on/off
If set to on, force the input filename to be regarded as an @sc{html}
document---the same as @samp{-F}.

@item ftp_proxy = @var{string}
Use @var{string} as @sc{ftp} proxy, instead of the one specified in
environment.

@item glob = on/off
Turn globbing on/off---the same as @samp{-g}.

@item header = @var{string}
Define an additional header, like @samp{--header}.

@item html_extension = on/off
Add a @samp{.html} extension to @samp{text/html} files without it, like
@samp{-E}.

@item http_passwd = @var{string}
Set @sc{http} password.

@item http_proxy = @var{string}
Use @var{string} as @sc{http} proxy, instead of the one specified in
environment.

@item http_user = @var{string}
Set @sc{http} user to @var{string}.

@item ignore_length = on/off
When set to on, ignore @code{Content-Length} header; the same as
@samp{--ignore-length}.

@item ignore_tags = @var{string}
Ignore certain HTML tags when doing a recursive retrieval, just like
@samp{-G} / @samp{--ignore-tags}.

@item include_directories = @var{string}
Specify a comma-separated list of directories you wish to follow when
downloading---the same as @samp{-I}.

@item input = @var{string}
Read the @sc{url}s from @var{string}, like @samp{-i}.

@item kill_longer = on/off
Consider data longer than specified in content-length header as invalid
(and retry getting it).  The default behaviour is to save as much data
as there is, provided there is more than or equal to the value in
@code{Content-Length}.

@item limit_rate = @var{rate}
Limit the download speed to no more than @var{rate} bytes per second.
The same as @samp{--limit-rate}.

@item logfile = @var{string}
Set logfile---the same as @samp{-o}.

@item login = @var{string}
Your user name on the remote machine, for @sc{ftp}.  Defaults to
@samp{anonymous}.

@item mirror = on/off
Turn mirroring on/off.  The same as @samp{-m}.

@item netrc = on/off
Turn reading netrc on or off.

@item noclobber = on/off
Same as @samp{-nc}.

@item no_parent = on/off
Disallow retrieving outside the directory hierarchy, like
@samp{--no-parent} (@pxref{Directory-Based Limits}).

@item no_proxy = @var{string}
Use @var{string} as the comma-separated list of domains to avoid in
proxy loading, instead of the one specified in environment.

@item output_document = @var{string}
Set the output filename---the same as @samp{-O}.

@item page_requisites = on/off
Download all ancillary documents necessary for a single HTML page to
display properly---the same as @samp{-p}.

@item passive_ftp = on/off/always/never
Set passive @sc{ftp}---the same as @samp{--passive-ftp}.  Some scripts
and @samp{.pm} (Perl module) files download files using @samp{wget
--passive-ftp}.  If your firewall does not allow this, you can set
@samp{passive_ftp = never} to override the commandline.

@item passwd = @var{string}
Set your @sc{ftp} password to @var{password}.  Without this setting, the
password defaults to @samp{username@@hostname.domainname}.

@item progress = @var{string}
Set the type of the progress indicator.  Legal types are ``dot'' and
``bar''.

@item proxy_user = @var{string}
Set proxy authentication user name to @var{string}, like @samp{--proxy-user}.

@item proxy_passwd = @var{string}
Set proxy authentication password to @var{string}, like @samp{--proxy-passwd}.

@item referer = @var{string}
Set HTTP @samp{Referer:} header just like @samp{--referer}.  (Note it
was the folks who wrote the @sc{http} spec who got the spelling of
``referrer'' wrong.)

@item quiet = on/off
Quiet mode---the same as @samp{-q}.

@item quota = @var{quota}
Specify the download quota, which is useful to put in the global
@file{wgetrc}.  When download quota is specified, Wget will stop
retrieving after the download sum has become greater than quota.  The
quota can be specified in bytes (default), kbytes @samp{k} appended) or
mbytes (@samp{m} appended).  Thus @samp{quota = 5m} will set the quota
to 5 mbytes.  Note that the user's startup file overrides system
settings.

@item reclevel = @var{n}
Recursion level---the same as @samp{-l}.

@item recursive = on/off
Recursive on/off---the same as @samp{-r}.

@item relative_only = on/off
Follow only relative links---the same as @samp{-L} (@pxref{Relative
Links}).

@item remove_listing = on/off
If set to on, remove @sc{ftp} listings downloaded by Wget.  Setting it
to off is the same as @samp{-nr}.

@item retr_symlinks = on/off
When set to on, retrieve symbolic links as if they were plain files; the
same as @samp{--retr-symlinks}.

@item robots = on/off
Specify whether the norobots convention is respected by Wget, ``on'' by
default.  This switch controls both the @file{/robots.txt} and the
@samp{nofollow} aspect of the spec.  @xref{Robot Exclusion}, for more
details about this.  Be sure you know what you are doing before turning
this off.

@item server_response = on/off
Choose whether or not to print the @sc{http} and @sc{ftp} server
responses---the same as @samp{-S}.

@item span_hosts = on/off
Same as @samp{-H}.

@item timeout = @var{n}
Set timeout value---the same as @samp{-T}.

@item timestamping = on/off
Turn timestamping on/off.  The same as @samp{-N} (@pxref{Time-Stamping}).

@item tries = @var{n}
Set number of retries per @sc{url}---the same as @samp{-t}.

@item use_proxy = on/off
Turn proxy support on/off.  The same as @samp{-Y}.

@item verbose = on/off
Turn verbose on/off---the same as @samp{-v}/@samp{-nv}.

@item wait = @var{n}
Wait @var{n} seconds between retrievals---the same as @samp{-w}.

@item waitretry = @var{n}
Wait up to @var{n} seconds between retries of failed retrievals
only---the same as @samp{--waitretry}.  Note that this is turned on by
default in the global @file{wgetrc}.

@item randomwait = on/off
Turn random between-request wait times on or off. The same as 
@samp{--random-wait}.
@end table

@node Sample Wgetrc,  , Wgetrc Commands, Startup File
@section Sample Wgetrc
@cindex sample wgetrc

This is the sample initialization file, as given in the distribution.
It is divided in two section---one for global usage (suitable for global
startup file), and one for local usage (suitable for
@file{$HOME/.wgetrc}).  Be careful about the things you change.

Note that almost all the lines are commented out.  For a command to have
any effect, you must remove the @samp{#} character at the beginning of
its line.

@example
@include sample.wgetrc.munged_for_texi_inclusion
@end example

@node Examples, Various, Startup File, Top
@chapter Examples
@cindex examples

@c man begin EXAMPLES
The examples are divided into three sections loosely based on their
complexity.

@menu
* Simple Usage::         Simple, basic usage of the program.
* Advanced Usage::       Advanced tips.
* Very Advanced Usage::  The hairy stuff.
@end menu

@node Simple Usage, Advanced Usage, Examples, Examples
@section Simple Usage

@itemize @bullet
@item
Say you want to download a @sc{url}.  Just type:

@example
wget http://fly.srk.fer.hr/
@end example

@item
But what will happen if the connection is slow, and the file is lengthy?
The connection will probably fail before the whole file is retrieved,
more than once.  In this case, Wget will try getting the file until it
either gets the whole of it, or exceeds the default number of retries
(this being 20).  It is easy to change the number of tries to 45, to
insure that the whole file will arrive safely:

@example
wget --tries=45 http://fly.srk.fer.hr/jpg/flyweb.jpg
@end example

@item
Now let's leave Wget to work in the background, and write its progress
to log file @file{log}.  It is tiring to type @samp{--tries}, so we
shall use @samp{-t}.

@example
wget -t 45 -o log http://fly.srk.fer.hr/jpg/flyweb.jpg &
@end example

The ampersand at the end of the line makes sure that Wget works in the
background.  To unlimit the number of retries, use @samp{-t inf}.

@item
The usage of @sc{ftp} is as simple.  Wget will take care of login and
password.

@example
wget ftp://gnjilux.srk.fer.hr/welcome.msg
@end example

@item
If you specify a directory, Wget will retrieve the directory listing,
parse it and convert it to @sc{html}.  Try:

@example
wget ftp://prep.ai.mit.edu/pub/gnu/
links index.html
@end example
@end itemize

@node Advanced Usage, Very Advanced Usage, Simple Usage, Examples
@section Advanced Usage

@itemize @bullet
@item
You have a file that contains the URLs you want to download?  Use the
@samp{-i} switch:

@example
wget -i @var{file}
@end example

If you specify @samp{-} as file name, the @sc{url}s will be read from
standard input.

@item
Create a five levels deep mirror image of the GNU web site, with the
same directory structure the original has, with only one try per
document, saving the log of the activities to @file{gnulog}:

@example
wget -r http://www.gnu.org/ -o gnulog
@end example

@item
The same as the above, but convert the links in the @sc{html} files to
point to local files, so you can view the documents off-line:

@example
wget --convert-links -r http://www.gnu.org/ -o gnulog
@end example

@item
Retrieve only one HTML page, but make sure that all the elements needed
for the page to be displayed, such as inline images and external style
sheets, are also downloaded.  Also make sure the downloaded page
references the downloaded links.

@example
wget -p --convert-links http://www.server.com/dir/page.html
@end example

The HTML page will be saved to @file{www.server.com/dir/page.html}, and
the images, stylesheets, etc., somewhere under @file{www.server.com/},
depending on where they were on the remote server.

@item
The same as the above, but without the @file{www.server.com/} directory.
In fact, I don't want to have all those random server directories
anyway---just save @emph{all} those files under a @file{download/}
subdirectory of the current directory.

@example
wget -p --convert-links -nH -nd -Pdownload \
     http://www.server.com/dir/page.html
@end example

@item
Retrieve the index.html of @samp{www.lycos.com}, showing the original
server headers:

@example
wget -S http://www.lycos.com/
@end example

@item
Save the server headers with the file, perhaps for post-processing.

@example
wget -s http://www.lycos.com/
more index.html
@end example

@item
Retrieve the first two levels of @samp{wuarchive.wustl.edu}, saving them
to @file{/tmp}.

@example
wget -r -l2 -P/tmp ftp://wuarchive.wustl.edu/
@end example

@item
You want to download all the @sc{gif}s from a directory on an @sc{http}
server.  You tried @samp{wget http://www.server.com/dir/*.gif}, but that
didn't work because @sc{http} retrieval does not support globbing.  In
that case, use:

@example
wget -r -l1 --no-parent -A.gif http://www.server.com/dir/
@end example

More verbose, but the effect is the same.  @samp{-r -l1} means to
retrieve recursively (@pxref{Recursive Retrieval}), with maximum depth
of 1.  @samp{--no-parent} means that references to the parent directory
are ignored (@pxref{Directory-Based Limits}), and @samp{-A.gif} means to
download only the @sc{gif} files.  @samp{-A "*.gif"} would have worked
too.

@item
Suppose you were in the middle of downloading, when Wget was
interrupted.  Now you do not want to clobber the files already present.
It would be:

@example
wget -nc -r http://www.gnu.org/
@end example

@item
If you want to encode your own username and password to @sc{http} or
@sc{ftp}, use the appropriate @sc{url} syntax (@pxref{URL Format}).

@example
wget ftp://hniksic:mypassword@@unix.server.com/.emacs
@end example

Note, however, that this usage is not advisable on multi-user systems
because it reveals your password to anyone who looks at the output of
@code{ps}.

@cindex redirecting output
@item
You would like the output documents to go to standard output instead of
to files?

@example
wget -O - http://jagor.srce.hr/ http://www.srce.hr/
@end example

You can also combine the two options and make pipelines to retrieve the
documents from remote hotlists:

@example
wget -O - http://cool.list.com/ | wget --force-html -i -
@end example
@end itemize

@node Very Advanced Usage,  , Advanced Usage, Examples
@section Very Advanced Usage

@cindex mirroring
@itemize @bullet
@item
If you wish Wget to keep a mirror of a page (or @sc{ftp}
subdirectories), use @samp{--mirror} (@samp{-m}), which is the shorthand
for @samp{-r -l inf -N}.  You can put Wget in the crontab file asking it
to recheck a site each Sunday:

@example
crontab
0 0 * * 0 wget --mirror http://www.gnu.org/ -o /home/me/weeklog
@end example

@item
In addition to the above, you want the links to be converted for local
viewing.  But, after having read this manual, you know that link
conversion doesn't play well with timestamping, so you also want Wget to
back up the original HTML files before the conversion.  Wget invocation
would look like this:

@example
wget --mirror --convert-links --backup-converted  \
     http://www.gnu.org/ -o /home/me/weeklog
@end example

@item
But you've also noticed that local viewing doesn't work all that well
when HTML files are saved under extensions other than @samp{.html},
perhaps because they were served as @file{index.cgi}.  So you'd like
Wget to rename all the files served with content-type @samp{text/html}
to @file{@var{name}.html}.

@example
wget --mirror --convert-links --backup-converted \
     --html-extension -o /home/me/weeklog        \
     http://www.gnu.org/
@end example

Or, with less typing:

@example
wget -m -k -K -E http://www.gnu.org/ -o /home/me/weeklog
@end example
@end itemize
@c man end

@node Various, Appendices, Examples, Top
@chapter Various
@cindex various

This chapter contains all the stuff that could not fit anywhere else.

@menu
* Proxies::             Support for proxy servers
* Distribution::        Getting the latest version.
* Mailing List::        Wget mailing list for announcements and discussion.
* Reporting Bugs::      How and where to report bugs.
* Portability::         The systems Wget works on.
* Signals::             Signal-handling performed by Wget.
@end menu

@node Proxies, Distribution, Various, Various
@section Proxies
@cindex proxies

@dfn{Proxies} are special-purpose @sc{http} servers designed to transfer
data from remote servers to local clients.  One typical use of proxies
is lightening network load for users behind a slow connection.  This is
achieved by channeling all @sc{http} and @sc{ftp} requests through the
proxy which caches the transferred data.  When a cached resource is
requested again, proxy will return the data from cache.  Another use for
proxies is for companies that separate (for security reasons) their
internal networks from the rest of Internet.  In order to obtain
information from the Web, their users connect and retrieve remote data
using an authorized proxy.

Wget supports proxies for both @sc{http} and @sc{ftp} retrievals.  The
standard way to specify proxy location, which Wget recognizes, is using
the following environment variables:

@table @code
@item http_proxy
This variable should contain the @sc{url} of the proxy for @sc{http}
connections.

@item ftp_proxy
This variable should contain the @sc{url} of the proxy for @sc{ftp}
connections.  It is quite common that @sc{http_proxy} and @sc{ftp_proxy}
are set to the same @sc{url}.

@item no_proxy
This variable should contain a comma-separated list of domain extensions
proxy should @emph{not} be used for.  For instance, if the value of
@code{no_proxy} is @samp{.mit.edu}, proxy will not be used to retrieve
documents from MIT.
@end table

In addition to the environment variables, proxy location and settings
may be specified from within Wget itself.

@table @samp
@item -Y on/off
@itemx --proxy=on/off
@itemx proxy = on/off
This option may be used to turn the proxy support on or off.  Proxy
support is on by default, provided that the appropriate environment
variables are set.

@item http_proxy = @var{URL}
@itemx ftp_proxy = @var{URL}
@itemx no_proxy = @var{string}
These startup file variables allow you to override the proxy settings
specified by the environment.
@end table

Some proxy servers require authorization to enable you to use them.  The
authorization consists of @dfn{username} and @dfn{password}, which must
be sent by Wget.  As with @sc{http} authorization, several
authentication schemes exist.  For proxy authorization only the
@code{Basic} authentication scheme is currently implemented.

You may specify your username and password either through the proxy
@sc{url} or through the command-line options.  Assuming that the
company's proxy is located at @samp{proxy.company.com} at port 8001, a
proxy @sc{url} location containing authorization data might look like
this:

@example
http://hniksic:mypassword@@proxy.company.com:8001/
@end example

Alternatively, you may use the @samp{proxy-user} and
@samp{proxy-password} options, and the equivalent @file{.wgetrc}
settings @code{proxy_user} and @code{proxy_passwd} to set the proxy
username and password.

@node Distribution, Mailing List, Proxies, Various
@section Distribution
@cindex latest version

Like all GNU utilities, the latest version of Wget can be found at the
master GNU archive site prep.ai.mit.edu, and its mirrors.  For example,
Wget @value{VERSION} can be found at
@url{ftp://prep.ai.mit.edu/gnu/wget/wget-@value{VERSION}.tar.gz}

@node Mailing List, Reporting Bugs, Distribution, Various
@section Mailing List
@cindex mailing list
@cindex list

Wget has its own mailing list at @email{wget@@sunsite.dk}, thanks
to Karsten Thygesen.  The mailing list is for discussion of Wget
features and web, reporting Wget bugs (those that you think may be of
interest to the public) and mailing announcements.  You are welcome to
subscribe.  The more people on the list, the better!

To subscribe, send mail to @email{wget-subscribe@@sunsite.dk}.
the magic word @samp{subscribe} in the subject line.  Unsubscribe by
mailing to @email{wget-unsubscribe@@sunsite.dk}.

The mailing list is archived at @url{http://fly.srk.fer.hr/archive/wget}.
Alternative archive is available at
@url{http://www.mail-archive.com/wget%40sunsite.auc.dk/}.
 
@node Reporting Bugs, Portability, Mailing List, Various
@section Reporting Bugs
@cindex bugs
@cindex reporting bugs
@cindex bug reports

@c man begin BUGS
You are welcome to send bug reports about GNU Wget to
@email{bug-wget@@gnu.org}.

Before actually submitting a bug report, please try to follow a few
simple guidelines.

@enumerate
@item
Please try to ascertain that the behaviour you see really is a bug.  If
Wget crashes, it's a bug.  If Wget does not behave as documented,
it's a bug.  If things work strange, but you are not sure about the way
they are supposed to work, it might well be a bug.

@item
Try to repeat the bug in as simple circumstances as possible.  E.g. if
Wget crashes while downloading @samp{wget -rl0 -kKE -t5 -Y0
http://yoyodyne.com -o /tmp/log}, you should try to see if the crash is
repeatable, and if will occur with a simpler set of options.  You might
even try to start the download at the page where the crash occurred to
see if that page somehow triggered the crash.

Also, while I will probably be interested to know the contents of your
@file{.wgetrc} file, just dumping it into the debug message is probably
a bad idea.  Instead, you should first try to see if the bug repeats
with @file{.wgetrc} moved out of the way.  Only if it turns out that
@file{.wgetrc} settings affect the bug, mail me the relevant parts of
the file.

@item
Please start Wget with @samp{-d} option and send the log (or the
relevant parts of it).  If Wget was compiled without debug support,
recompile it.  It is @emph{much} easier to trace bugs with debug support
on.

@item
If Wget has crashed, try to run it in a debugger, e.g. @code{gdb `which
wget` core} and type @code{where} to get the backtrace.
@end enumerate
@c man end

@node Portability, Signals, Reporting Bugs, Various
@section Portability
@cindex portability
@cindex operating systems

Since Wget uses GNU Autoconf for building and configuring, and avoids
using ``special'' ultra--mega--cool features of any particular Unix, it
should compile (and work) on all common Unix flavors.

Various Wget versions have been compiled and tested under many kinds of
Unix systems, including Solaris, Linux, SunOS, OSF (aka Digital Unix),
Ultrix, *BSD, IRIX, and others; refer to the file @file{MACHINES} in the
distribution directory for a comprehensive list.  If you compile it on
an architecture not listed there, please let me know so I can update it.

Wget should also compile on the other Unix systems, not listed in
@file{MACHINES}.  If it doesn't, please let me know.

Thanks to kind contributors, this version of Wget compiles and works on
Microsoft Windows 95 and Windows NT platforms.  It has been compiled
successfully using MS Visual C++ 4.0, Watcom, and Borland C compilers,
with Winsock as networking software.  Naturally, it is crippled of some
features available on Unix, but it should work as a substitute for
people stuck with Windows.  Note that the Windows port is
@strong{neither tested nor maintained} by me---all questions and
problems should be reported to Wget mailing list at
@email{wget@@sunsite.dk} where the maintainers will look at them.

@node Signals,  , Portability, Various
@section Signals
@cindex signal handling
@cindex hangup

Since the purpose of Wget is background work, it catches the hangup
signal (@code{SIGHUP}) and ignores it.  If the output was on standard
output, it will be redirected to a file named @file{wget-log}.
Otherwise, @code{SIGHUP} is ignored.  This is convenient when you wish
to redirect the output of Wget after having started it.

@example
$ wget http://www.ifi.uio.no/~larsi/gnus.tar.gz &
$ kill -HUP %%     # Redirect the output to wget-log
@end example

Other than that, Wget will not try to interfere with signals in any way.
@kbd{C-c}, @code{kill -TERM} and @code{kill -KILL} should kill it alike.

@node Appendices, Copying, Various, Top
@chapter Appendices

This chapter contains some references I consider useful.

@menu
* Robot Exclusion::         Wget's support for RES.
* Security Considerations:: Security with Wget.
* Contributors::            People who helped.
@end menu

@node Robot Exclusion, Security Considerations, Appendices, Appendices
@section Robot Exclusion
@cindex robot exclusion
@cindex robots.txt
@cindex server maintenance

It is extremely easy to make Wget wander aimlessly around a web site,
sucking all the available data in progress.  @samp{wget -r @var{site}},
and you're set.  Great?  Not for the server admin.

As long as Wget is only retrieving static pages, and doing it at a
reasonable rate (see the @samp{--wait} option), there's not much of a
problem.  The trouble is that Wget can't tell the difference between the
smallest static page and the most demanding CGI.  A site I know has a
section handled by an, uh, @dfn{bitchin'} CGI Perl script that converts
Info files to HTML on the fly.  The script is slow, but works well
enough for human users viewing an occasional Info file.  However, when
someone's recursive Wget download stumbles upon the index page that
links to all the Info files through the script, the system is brought to
its knees without providing anything useful to the downloader.

To avoid this kind of accident, as well as to preserve privacy for
documents that need to be protected from well-behaved robots, the
concept of @dfn{robot exclusion} has been invented.  The idea is that
the server administrators and document authors can specify which
portions of the site they wish to protect from the robots.

The most popular mechanism, and the de facto standard supported by all
the major robots, is the ``Robots Exclusion Standard'' (RES) written by
Martijn Koster et al. in 1994.  It specifies the format of a text file
containing directives that instruct the robots which URL paths to avoid.
To be found by the robots, the specifications must be placed in
@file{/robots.txt} in the server root, which the robots are supposed to
download and parse.

Although Wget is not a web robot in the strictest sense of the word, it
can downloads large parts of the site without the user's intervention to
download an individual page.  Because of that, Wget honors RES when
downloading recursively.  For instance, when you issue:

@example
wget -r http://www.server.com/
@end example

First the index of @samp{www.server.com} will be downloaded.  If Wget
finds that it wants to download more documents from that server, it will
request @samp{http://www.server.com/robots.txt} and, if found, use it
for further downloads.  @file{robots.txt} is loaded only once per each
server.

Until version 1.8, Wget supported the first version of the standard,
written by Martijn Koster in 1994 and available at
@url{http://www.robotstxt.org/wc/norobots.html}.  As of version 1.8,
Wget has supported the additional directives specified in the internet
draft @samp{<draft-koster-robots-00.txt>} titled ``A Method for Web
Robots Control''.  The draft, which has as far as I know never made to
an @sc{rfc}, is available at
@url{http://www.robotstxt.org/wc/norobots-rfc.txt}.

This manual no longer includes the text of the Robot Exclusion Standard.

The second, less known mechanism, enables the author of an individual
document to specify whether they want the links from the file to be
followed by a robot.  This is achieved using the @code{META} tag, like
this:

@example
<meta name="robots" content="nofollow">
@end example

This is explained in some detail at
@url{http://www.robotstxt.org/wc/meta-user.html}.  Wget supports this
method of robot exclusion in addition to the usual @file{/robots.txt}
exclusion.

If you know what you are doing and really really wish to turn off the
robot exclusion, set the @code{robots} variable to @samp{off} in your
@file{.wgetrc}.  You can achieve the same effect from the command line
using the @code{-e} switch, e.g. @samp{wget -e robots=off @var{url}...}.

@node Security Considerations, Contributors, Robot Exclusion, Appendices
@section Security Considerations
@cindex security

When using Wget, you must be aware that it sends unencrypted passwords
through the network, which may present a security problem.  Here are the
main issues, and some solutions.

@enumerate
@item The passwords on the command line are visible using @code{ps}.
The best way around it is to use @code{wget -i -} and feed the @sc{url}s
to Wget's standard input, each on a separate line, terminated by
@kbd{C-d}.  Another workaround is to use @file{.netrc} to store
passwords; however, storing unencrypted passwords is also considered a
security risk.

@item
Using the insecure @dfn{basic} authentication scheme, unencrypted
passwords are transmitted through the network routers and gateways.

@item
The @sc{ftp} passwords are also in no way encrypted.  There is no good
solution for this at the moment.

@item
Although the ``normal'' output of Wget tries to hide the passwords,
debugging logs show them, in all forms.  This problem is avoided by
being careful when you send debug logs (yes, even when you send them to
me).
@end enumerate

@node Contributors,  , Security Considerations, Appendices
@section Contributors
@cindex contributors

@iftex
GNU Wget was written by Hrvoje Nik@v{s}i@'{c} @email{hniksic@@arsdigita.com}.
@end iftex
@ifinfo
GNU Wget was written by Hrvoje Niksic @email{hniksic@@arsdigita.com}.
@end ifinfo
However, its development could never have gone as far as it has, were it
not for the help of many people, either with bug reports, feature
proposals, patches, or letters saying ``Thanks!''.

Special thanks goes to the following people (no particular order):

@itemize @bullet
@item
Karsten Thygesen---donated system resources such as the mailing list,
web space, and @sc{ftp} space, along with a lot of time to make these
actually work.

@item
Shawn McHorse---bug reports and patches.

@item
Kaveh R. Ghazi---on-the-fly @code{ansi2knr}-ization.  Lots of
portability fixes.

@item
Gordon Matzigkeit---@file{.netrc} support.

@item
@iftex
Zlatko @v{C}alu@v{s}i@'{c}, Tomislav Vujec and Dra@v{z}en
Ka@v{c}ar---feature suggestions and ``philosophical'' discussions.
@end iftex
@ifinfo
Zlatko Calusic, Tomislav Vujec and Drazen Kacar---feature suggestions
and ``philosophical'' discussions.
@end ifinfo

@item
Darko Budor---initial port to Windows.

@item
Antonio Rosella---help and suggestions, plus the Italian translation.

@item
@iftex
Tomislav Petrovi@'{c}, Mario Miko@v{c}evi@'{c}---many bug reports and
suggestions.
@end iftex
@ifinfo
Tomislav Petrovic, Mario Mikocevic---many bug reports and suggestions.
@end ifinfo

@item
@iftex
Fran@,{c}ois Pinard---many thorough bug reports and discussions.
@end iftex
@ifinfo
Francois Pinard---many thorough bug reports and discussions.
@end ifinfo

@item
Karl Eichwalder---lots of help with internationalization and other
things.

@item
Junio Hamano---donated support for Opie and @sc{http} @code{Digest}
authentication.

@item
The people who provided donations for development, including Brian
Gough.
@end itemize

The following people have provided patches, bug/build reports, useful
suggestions, beta testing services, fan mail and all the other things
that make maintenance so much fun:

Ian Abbott
Tim Adam,
Adrian Aichner,
Martin Baehr,
Dieter Baron,
Roger Beeman,
Dan Berger,
T. Bharath,
Paul Bludov,
Daniel Bodea,
Mark Boyns,
John Burden,
Wanderlei Cavassin,
Gilles Cedoc,
Tim Charron,
Noel Cragg,
@iftex
Kristijan @v{C}onka@v{s},
@end iftex
@ifinfo
Kristijan Conkas,
@end ifinfo
John Daily,
Andrew Davison,
Andrew Deryabin,
Ulrich Drepper,
Marc Duponcheel,
@iftex
Damir D@v{z}eko,
@end iftex
@ifinfo
Damir Dzeko,
@end ifinfo
Alan Eldridge,
@iftex
Aleksandar Erkalovi@'{c},
@end iftex
@ifinfo
Aleksandar Erkalovic,
@end ifinfo
Andy Eskilsson,
Christian Fraenkel,
Masashi Fujita,
Howard Gayle,
Marcel Gerrits,
Lemble Gregory,
Hans Grobler,
Mathieu Guillaume,
Dan Harkless,
Herold Heiko,
Jochen Hein,
Karl Heuer,
HIROSE Masaaki,
Gregor Hoffleit,
Erik Magnus Hulthen,
Richard Huveneers,
Jonas Jensen,
Simon Josefsson,
@iftex
Mario Juri@'{c},
@end iftex
@ifinfo
Mario Juric,
@end ifinfo
@iftex
Hack Kampbj@o rn,
@end iftex
@ifinfo
Hack Kampbjorn,
@end ifinfo
Const Kaplinsky,
@iftex
Goran Kezunovi@'{c},
@end iftex
@ifinfo
Goran Kezunovic,
@end ifinfo
Robert Kleine,
KOJIMA Haime,
Fila Kolodny,
Alexander Kourakos,
Martin Kraemer,
@tex
$\Sigma\acute{\iota}\mu o\varsigma\;
\Xi\varepsilon\nu\iota\tau\acute{\epsilon}\lambda\lambda\eta\varsigma$
(Simos KSenitellis),
@end tex
@ifinfo
Simos KSenitellis,
@end ifinfo
Hrvoje Lacko,
Daniel S. Lewart,
@iftex
Nicol@'{a}s Lichtmeier,
@end iftex
@ifinfo
Nicolas Lichtmeier,
@end ifinfo
Dave Love,
Alexander V. Lukyanov,
Jordan Mendelson,
Lin Zhe Min,
Tim Mooney,
Simon Munton,
Charlie Negyesi,
R. K. Owen,
Andrew Pollock,
Steve Pothier,
@iftex
Jan P@v{r}ikryl,
@end iftex
@ifinfo
Jan Prikryl,
@end ifinfo
Marin Purgar,
@iftex
Csaba R@'{a}duly,
@end iftex
@ifinfo
Csaba Raduly,
@end ifinfo
Keith Refson,
Tyler Riddle,
Tobias Ringstrom,
@c Texinfo doesn't grok @'{@i}, so we have to use TeX itself.
@tex
Juan Jos\'{e} Rodr\'{\i}gues,
@end tex
@ifinfo
Juan Jose Rodrigues,
@end ifinfo
Edward J. Sabol,
Heinz Salzmann,
Robert Schmidt,
Andreas Schwab,
Chris Seawood,
Toomas Soome,
Tage Stabell-Kulo,
Sven Sternberger,
Markus Strasser,
John Summerfield,
Szakacsits Szabolcs,
Mike Thomas,
Philipp Thomas,
Dave Turner,
Russell Vincent,
Charles G Waldman,
Douglas E. Wegscheid,
Jasmin Zainul,
@iftex
Bojan @v{Z}drnja,
@end iftex
@ifinfo
Bojan Zdrnja,
@end ifinfo
Kristijan Zimmer.

Apologies to all who I accidentally left out, and many thanks to all the
subscribers of the Wget mailing list.

@node Copying, Concept Index, Appendices, Top
@chapter Copying
@cindex copying
@cindex GPL
@cindex GFDL
@cindex free software

GNU Wget is licensed under the GNU GPL, which makes it @dfn{free
software}.

Please note that ``free'' in ``free software'' refers to liberty, not
price.  As some GNU project advocates like to point out, think of ``free
speech'' rather than ``free beer''.  The exact and legally binding
distribution terms are spelled out below; in short, you have the right
(freedom) to run and change Wget and distribute it to other people, and
even---if you want---charge money for doing either.  The important
restriction is that you have to grant your recipients the same rights
and impose the same restrictions.

This method of licensing software is also known as @dfn{open source}
because, among other things, it makes sure that all recipients will
receive the source code along with the program, and be able to improve
it.  The GNU project prefers the term ``free software'' for reasons
outlined at
@url{http://www.gnu.org/philosophy/free-software-for-freedom.html}.

The exact license terms are defined by this paragraph and the GNU
General Public License it refers to:

@quotation
GNU Wget is free software; you can redistribute it and/or modify it
under the terms of the GNU General Public License as published by the
Free Software Foundation; either version 2 of the License, or (at your
option) any later version.

GNU Wget is distributed in the hope that it will be useful, but WITHOUT
ANY WARRANTY; without even the implied warranty of MERCHANTABILITY or
FITNESS FOR A PARTICULAR PURPOSE.  See the GNU General Public License
for more details.

A copy of the GNU General Public License is included as part of this
manual; if you did not receive it, write to the Free Software
Foundation, Inc., 675 Mass Ave, Cambridge, MA 02139, USA.
@end quotation

In addition to this, this manual is free in the same sense:

@quotation
Permission is granted to copy, distribute and/or modify this document
under the terms of the GNU Free Documentation License, Version 1.1 or
any later version published by the Free Software Foundation; with the
Invariant Sections being ``GNU General Public License'' and ``GNU Free
Documentation License'', with no Front-Cover Texts, and with no
Back-Cover Texts.  A copy of the license is included in the section
entitled ``GNU Free Documentation License''.
@end quotation

@c #### Maybe we should wrap these licenses in ifinfo?  Stallman says
@c that the GFDL needs to be present in the manual, and to me it would
@c suck to include the license for the manual and not the license for
@c the program.

The full texts of the GNU General Public License and of the GNU Free
Documentation License are available below.

@menu
* GNU General Public License::
* GNU Free Documentation License::
@end menu

@node GNU General Public License, GNU Free Documentation License, Copying, Copying
@section GNU General Public License
@center Version 2, June 1991

@display
Copyright @copyright{} 1989, 1991 Free Software Foundation, Inc.
675 Mass Ave, Cambridge, MA 02139, USA

Everyone is permitted to copy and distribute verbatim copies
of this license document, but changing it is not allowed.
@end display

@unnumberedsec Preamble

  The licenses for most software are designed to take away your
freedom to share and change it.  By contrast, the GNU General Public
License is intended to guarantee your freedom to share and change free
software---to make sure the software is free for all its users.  This
General Public License applies to most of the Free Software
Foundation's software and to any other program whose authors commit to
using it.  (Some other Free Software Foundation software is covered by
the GNU Library General Public License instead.)  You can apply it to
your programs, too.

  When we speak of free software, we are referring to freedom, not
price.  Our General Public Licenses are designed to make sure that you
have the freedom to distribute copies of free software (and charge for
this service if you wish), that you receive source code or can get it
if you want it, that you can change the software or use pieces of it
in new free programs; and that you know you can do these things.

  To protect your rights, we need to make restrictions that forbid
anyone to deny you these rights or to ask you to surrender the rights.
These restrictions translate to certain responsibilities for you if you
distribute copies of the software, or if you modify it.

  For example, if you distribute copies of such a program, whether
gratis or for a fee, you must give the recipients all the rights that
you have.  You must make sure that they, too, receive or can get the
source code.  And you must show them these terms so they know their
rights.

  We protect your rights with two steps: (1) copyright the software, and
(2) offer you this license which gives you legal permission to copy,
distribute and/or modify the software.

  Also, for each author's protection and ours, we want to make certain
that everyone understands that there is no warranty for this free
software.  If the software is modified by someone else and passed on, we
want its recipients to know that what they have is not the original, so
that any problems introduced by others will not reflect on the original
authors' reputations.

  Finally, any free program is threatened constantly by software
patents.  We wish to avoid the danger that redistributors of a free
program will individually obtain patent licenses, in effect making the
program proprietary.  To prevent this, we have made it clear that any
patent must be licensed for everyone's free use or not licensed at all.

  The precise terms and conditions for copying, distribution and
modification follow.

@iftex
@unnumberedsec TERMS AND CONDITIONS FOR COPYING, DISTRIBUTION AND MODIFICATION
@end iftex
@ifinfo
@center TERMS AND CONDITIONS FOR COPYING, DISTRIBUTION AND MODIFICATION
@end ifinfo

@enumerate
@item
This License applies to any program or other work which contains
a notice placed by the copyright holder saying it may be distributed
under the terms of this General Public License.  The ``Program'', below,
refers to any such program or work, and a ``work based on the Program''
means either the Program or any derivative work under copyright law:
that is to say, a work containing the Program or a portion of it,
either verbatim or with modifications and/or translated into another
language.  (Hereinafter, translation is included without limitation in
the term ``modification''.)  Each licensee is addressed as ``you''.

Activities other than copying, distribution and modification are not
covered by this License; they are outside its scope.  The act of
running the Program is not restricted, and the output from the Program
is covered only if its contents constitute a work based on the
Program (independent of having been made by running the Program).
Whether that is true depends on what the Program does.

@item
You may copy and distribute verbatim copies of the Program's
source code as you receive it, in any medium, provided that you
conspicuously and appropriately publish on each copy an appropriate
copyright notice and disclaimer of warranty; keep intact all the
notices that refer to this License and to the absence of any warranty;
and give any other recipients of the Program a copy of this License
along with the Program.

You may charge a fee for the physical act of transferring a copy, and
you may at your option offer warranty protection in exchange for a fee.

@item
You may modify your copy or copies of the Program or any portion
of it, thus forming a work based on the Program, and copy and
distribute such modifications or work under the terms of Section 1
above, provided that you also meet all of these conditions:

@enumerate a
@item
You must cause the modified files to carry prominent notices
stating that you changed the files and the date of any change.

@item
You must cause any work that you distribute or publish, that in
whole or in part contains or is derived from the Program or any
part thereof, to be licensed as a whole at no charge to all third
parties under the terms of this License.

@item
If the modified program normally reads commands interactively
when run, you must cause it, when started running for such
interactive use in the most ordinary way, to print or display an
announcement including an appropriate copyright notice and a
notice that there is no warranty (or else, saying that you provide
a warranty) and that users may redistribute the program under
these conditions, and telling the user how to view a copy of this
License.  (Exception: if the Program itself is interactive but
does not normally print such an announcement, your work based on
the Program is not required to print an announcement.)
@end enumerate

These requirements apply to the modified work as a whole.  If
identifiable sections of that work are not derived from the Program,
and can be reasonably considered independent and separate works in
themselves, then this License, and its terms, do not apply to those
sections when you distribute them as separate works.  But when you
distribute the same sections as part of a whole which is a work based
on the Program, the distribution of the whole must be on the terms of
this License, whose permissions for other licensees extend to the
entire whole, and thus to each and every part regardless of who wrote it.

Thus, it is not the intent of this section to claim rights or contest
your rights to work written entirely by you; rather, the intent is to
exercise the right to control the distribution of derivative or
collective works based on the Program.

In addition, mere aggregation of another work not based on the Program
with the Program (or with a work based on the Program) on a volume of
a storage or distribution medium does not bring the other work under
the scope of this License.

@item
You may copy and distribute the Program (or a work based on it,
under Section 2) in object code or executable form under the terms of
Sections 1 and 2 above provided that you also do one of the following:

@enumerate a
@item
Accompany it with the complete corresponding machine-readable
source code, which must be distributed under the terms of Sections
1 and 2 above on a medium customarily used for software interchange; or,

@item
Accompany it with a written offer, valid for at least three
years, to give any third party, for a charge no more than your
cost of physically performing source distribution, a complete
machine-readable copy of the corresponding source code, to be
distributed under the terms of Sections 1 and 2 above on a medium
customarily used for software interchange; or,

@item
Accompany it with the information you received as to the offer
to distribute corresponding source code.  (This alternative is
allowed only for noncommercial distribution and only if you
received the program in object code or executable form with such
an offer, in accord with Subsection b above.)
@end enumerate

The source code for a work means the preferred form of the work for
making modifications to it.  For an executable work, complete source
code means all the source code for all modules it contains, plus any
associated interface definition files, plus the scripts used to
control compilation and installation of the executable.  However, as a
special exception, the source code distributed need not include
anything that is normally distributed (in either source or binary
form) with the major components (compiler, kernel, and so on) of the
operating system on which the executable runs, unless that component
itself accompanies the executable.

If distribution of executable or object code is made by offering
access to copy from a designated place, then offering equivalent
access to copy the source code from the same place counts as
distribution of the source code, even though third parties are not
compelled to copy the source along with the object code.

@item
You may not copy, modify, sublicense, or distribute the Program
except as expressly provided under this License.  Any attempt
otherwise to copy, modify, sublicense or distribute the Program is
void, and will automatically terminate your rights under this License.
However, parties who have received copies, or rights, from you under
this License will not have their licenses terminated so long as such
parties remain in full compliance.

@item
You are not required to accept this License, since you have not
signed it.  However, nothing else grants you permission to modify or
distribute the Program or its derivative works.  These actions are
prohibited by law if you do not accept this License.  Therefore, by
modifying or distributing the Program (or any work based on the
Program), you indicate your acceptance of this License to do so, and
all its terms and conditions for copying, distributing or modifying
the Program or works based on it.

@item
Each time you redistribute the Program (or any work based on the
Program), the recipient automatically receives a license from the
original licensor to copy, distribute or modify the Program subject to
these terms and conditions.  You may not impose any further
restrictions on the recipients' exercise of the rights granted herein.
You are not responsible for enforcing compliance by third parties to
this License.

@item
If, as a consequence of a court judgment or allegation of patent
infringement or for any other reason (not limited to patent issues),
conditions are imposed on you (whether by court order, agreement or
otherwise) that contradict the conditions of this License, they do not
excuse you from the conditions of this License.  If you cannot
distribute so as to satisfy simultaneously your obligations under this
License and any other pertinent obligations, then as a consequence you
may not distribute the Program at all.  For example, if a patent
license would not permit royalty-free redistribution of the Program by
all those who receive copies directly or indirectly through you, then
the only way you could satisfy both it and this License would be to
refrain entirely from distribution of the Program.

If any portion of this section is held invalid or unenforceable under
any particular circumstance, the balance of the section is intended to
apply and the section as a whole is intended to apply in other
circumstances.

It is not the purpose of this section to induce you to infringe any
patents or other property right claims or to contest validity of any
such claims; this section has the sole purpose of protecting the
integrity of the free software distribution system, which is
implemented by public license practices.  Many people have made
generous contributions to the wide range of software distributed
through that system in reliance on consistent application of that
system; it is up to the author/donor to decide if he or she is willing
to distribute software through any other system and a licensee cannot
impose that choice.

This section is intended to make thoroughly clear what is believed to
be a consequence of the rest of this License.

@item
If the distribution and/or use of the Program is restricted in
certain countries either by patents or by copyrighted interfaces, the
original copyright holder who places the Program under this License
may add an explicit geographical distribution limitation excluding
those countries, so that distribution is permitted only in or among
countries not thus excluded.  In such case, this License incorporates
the limitation as if written in the body of this License.

@item
The Free Software Foundation may publish revised and/or new versions
of the General Public License from time to time.  Such new versions will
be similar in spirit to the present version, but may differ in detail to
address new problems or concerns.

Each version is given a distinguishing version number.  If the Program
specifies a version number of this License which applies to it and ``any
later version'', you have the option of following the terms and conditions
either of that version or of any later version published by the Free
Software Foundation.  If the Program does not specify a version number of
this License, you may choose any version ever published by the Free Software
Foundation.

@item
If you wish to incorporate parts of the Program into other free
programs whose distribution conditions are different, write to the author
to ask for permission.  For software which is copyrighted by the Free
Software Foundation, write to the Free Software Foundation; we sometimes
make exceptions for this.  Our decision will be guided by the two goals
of preserving the free status of all derivatives of our free software and
of promoting the sharing and reuse of software generally.

@iftex
@heading NO WARRANTY
@end iftex
@ifinfo
@center NO WARRANTY
@end ifinfo
@cindex no warranty

@item
BECAUSE THE PROGRAM IS LICENSED FREE OF CHARGE, THERE IS NO WARRANTY
FOR THE PROGRAM, TO THE EXTENT PERMITTED BY APPLICABLE LAW.  EXCEPT WHEN
OTHERWISE STATED IN WRITING THE COPYRIGHT HOLDERS AND/OR OTHER PARTIES
PROVIDE THE PROGRAM ``AS IS'' WITHOUT WARRANTY OF ANY KIND, EITHER EXPRESSED
OR IMPLIED, INCLUDING, BUT NOT LIMITED TO, THE IMPLIED WARRANTIES OF
MERCHANTABILITY AND FITNESS FOR A PARTICULAR PURPOSE.  THE ENTIRE RISK AS
TO THE QUALITY AND PERFORMANCE OF THE PROGRAM IS WITH YOU.  SHOULD THE
PROGRAM PROVE DEFECTIVE, YOU ASSUME THE COST OF ALL NECESSARY SERVICING,
REPAIR OR CORRECTION.

@item
IN NO EVENT UNLESS REQUIRED BY APPLICABLE LAW OR AGREED TO IN WRITING
WILL ANY COPYRIGHT HOLDER, OR ANY OTHER PARTY WHO MAY MODIFY AND/OR
REDISTRIBUTE THE PROGRAM AS PERMITTED ABOVE, BE LIABLE TO YOU FOR DAMAGES,
INCLUDING ANY GENERAL, SPECIAL, INCIDENTAL OR CONSEQUENTIAL DAMAGES ARISING
OUT OF THE USE OR INABILITY TO USE THE PROGRAM (INCLUDING BUT NOT LIMITED
TO LOSS OF DATA OR DATA BEING RENDERED INACCURATE OR LOSSES SUSTAINED BY
YOU OR THIRD PARTIES OR A FAILURE OF THE PROGRAM TO OPERATE WITH ANY OTHER
PROGRAMS), EVEN IF SUCH HOLDER OR OTHER PARTY HAS BEEN ADVISED OF THE
POSSIBILITY OF SUCH DAMAGES.
@end enumerate

@iftex
@heading END OF TERMS AND CONDITIONS
@end iftex
@ifinfo
@center END OF TERMS AND CONDITIONS
@end ifinfo

@page
@unnumberedsec How to Apply These Terms to Your New Programs

  If you develop a new program, and you want it to be of the greatest
possible use to the public, the best way to achieve this is to make it
free software which everyone can redistribute and change under these terms.

  To do so, attach the following notices to the program.  It is safest
to attach them to the start of each source file to most effectively
convey the exclusion of warranty; and each file should have at least
the ``copyright'' line and a pointer to where the full notice is found.

@smallexample
@var{one line to give the program's name and an idea of what it does.}
Copyright (C) 19@var{yy}  @var{name of author}

This program is free software; you can redistribute it and/or
modify it under the terms of the GNU General Public License
as published by the Free Software Foundation; either version 2
of the License, or (at your option) any later version.

This program is distributed in the hope that it will be useful,
but WITHOUT ANY WARRANTY; without even the implied warranty of
MERCHANTABILITY or FITNESS FOR A PARTICULAR PURPOSE.  See the
GNU General Public License for more details.

You should have received a copy of the GNU General Public License
along with this program; if not, write to the Free Software
Foundation, Inc., 675 Mass Ave, Cambridge, MA 02139, USA.
@end smallexample

Also add information on how to contact you by electronic and paper mail.

If the program is interactive, make it output a short notice like this
when it starts in an interactive mode:

@smallexample
Gnomovision version 69, Copyright (C) 19@var{yy} @var{name of author}
Gnomovision comes with ABSOLUTELY NO WARRANTY; for details
type `show w'.  This is free software, and you are welcome
to redistribute it under certain conditions; type `show c'
for details.
@end smallexample

The hypothetical commands @samp{show w} and @samp{show c} should show
the appropriate parts of the General Public License.  Of course, the
commands you use may be called something other than @samp{show w} and
@samp{show c}; they could even be mouse-clicks or menu items---whatever
suits your program.

You should also get your employer (if you work as a programmer) or your
school, if any, to sign a ``copyright disclaimer'' for the program, if
necessary.  Here is a sample; alter the names:

@smallexample
@group
Yoyodyne, Inc., hereby disclaims all copyright
interest in the program `Gnomovision'
(which makes passes at compilers) written
by James Hacker.

@var{signature of Ty Coon}, 1 April 1989
Ty Coon, President of Vice
@end group
@end smallexample

This General Public License does not permit incorporating your program into
proprietary programs.  If your program is a subroutine library, you may
consider it more useful to permit linking proprietary applications with the
library.  If this is what you want to do, use the GNU Library General
Public License instead of this License.

@node GNU Free Documentation License,  , GNU General Public License, Copying
@section GNU Free Documentation License
@center Version 1.1, March 2000

@display
Copyright (C) 2000  Free Software Foundation, Inc.
59 Temple Place, Suite 330, Boston, MA  02111-1307  USA

Everyone is permitted to copy and distribute verbatim copies
of this license document, but changing it is not allowed.
@end display
@sp 1
@enumerate 0
@item
PREAMBLE

The purpose of this License is to make a manual, textbook, or other
written document ``free'' in the sense of freedom: to assure everyone
the effective freedom to copy and redistribute it, with or without
modifying it, either commercially or noncommercially.  Secondarily,
this License preserves for the author and publisher a way to get
credit for their work, while not being considered responsible for
modifications made by others.

This License is a kind of ``copyleft'', which means that derivative
works of the document must themselves be free in the same sense.  It
complements the GNU General Public License, which is a copyleft
license designed for free software.

We have designed this License in order to use it for manuals for free
software, because free software needs free documentation: a free
program should come with manuals providing the same freedoms that the
software does.  But this License is not limited to software manuals;
it can be used for any textual work, regardless of subject matter or
whether it is published as a printed book.  We recommend this License
principally for works whose purpose is instruction or reference.

@sp 1
@item
APPLICABILITY AND DEFINITIONS

This License applies to any manual or other work that contains a
notice placed by the copyright holder saying it can be distributed
under the terms of this License.  The ``Document'', below, refers to any
such manual or work.  Any member of the public is a licensee, and is
addressed as ``you''.

A ``Modified Version'' of the Document means any work containing the
Document or a portion of it, either copied verbatim, or with
modifications and/or translated into another language.

A ``Secondary Section'' is a named appendix or a front-matter section of
the Document that deals exclusively with the relationship of the
publishers or authors of the Document to the Document's overall subject
(or to related matters) and contains nothing that could fall directly
within that overall subject.  (For example, if the Document is in part a
textbook of mathematics, a Secondary Section may not explain any
mathematics.)  The relationship could be a matter of historical
connection with the subject or with related matters, or of legal,
commercial, philosophical, ethical or political position regarding
them.

The ``Invariant Sections'' are certain Secondary Sections whose titles
are designated, as being those of Invariant Sections, in the notice
that says that the Document is released under this License.

The ``Cover Texts'' are certain short passages of text that are listed,
as Front-Cover Texts or Back-Cover Texts, in the notice that says that
the Document is released under this License.

A ``Transparent'' copy of the Document means a machine-readable copy,
represented in a format whose specification is available to the
general public, whose contents can be viewed and edited directly and
straightforwardly with generic text editors or (for images composed of
pixels) generic paint programs or (for drawings) some widely available
drawing editor, and that is suitable for input to text formatters or
for automatic translation to a variety of formats suitable for input
to text formatters.  A copy made in an otherwise Transparent file
format whose markup has been designed to thwart or discourage
subsequent modification by readers is not Transparent.  A copy that is
not ``Transparent'' is called ``Opaque''.

Examples of suitable formats for Transparent copies include plain
ASCII without markup, Texinfo input format, LaTeX input format, SGML
or XML using a publicly available DTD, and standard-conforming simple
HTML designed for human modification.  Opaque formats include
PostScript, PDF, proprietary formats that can be read and edited only
by proprietary word processors, SGML or XML for which the DTD and/or
processing tools are not generally available, and the
machine-generated HTML produced by some word processors for output
purposes only.

The ``Title Page'' means, for a printed book, the title page itself,
plus such following pages as are needed to hold, legibly, the material
this License requires to appear in the title page.  For works in
formats which do not have any title page as such, ``Title Page'' means
the text near the most prominent appearance of the work's title,
preceding the beginning of the body of the text.
@sp 1
@item
VERBATIM COPYING

You may copy and distribute the Document in any medium, either
commercially or noncommercially, provided that this License, the
copyright notices, and the license notice saying this License applies
to the Document are reproduced in all copies, and that you add no other
conditions whatsoever to those of this License.  You may not use
technical measures to obstruct or control the reading or further
copying of the copies you make or distribute.  However, you may accept
compensation in exchange for copies.  If you distribute a large enough
number of copies you must also follow the conditions in section 3.

You may also lend copies, under the same conditions stated above, and
you may publicly display copies.
@sp 1
@item
COPYING IN QUANTITY

If you publish printed copies of the Document numbering more than 100,
and the Document's license notice requires Cover Texts, you must enclose
the copies in covers that carry, clearly and legibly, all these Cover
Texts: Front-Cover Texts on the front cover, and Back-Cover Texts on
the back cover.  Both covers must also clearly and legibly identify
you as the publisher of these copies.  The front cover must present
the full title with all words of the title equally prominent and
visible.  You may add other material on the covers in addition.
Copying with changes limited to the covers, as long as they preserve
the title of the Document and satisfy these conditions, can be treated
as verbatim copying in other respects.

If the required texts for either cover are too voluminous to fit
legibly, you should put the first ones listed (as many as fit
reasonably) on the actual cover, and continue the rest onto adjacent
pages.

If you publish or distribute Opaque copies of the Document numbering
more than 100, you must either include a machine-readable Transparent
copy along with each Opaque copy, or state in or with each Opaque copy
a publicly-accessible computer-network location containing a complete
Transparent copy of the Document, free of added material, which the
general network-using public has access to download anonymously at no
charge using public-standard network protocols.  If you use the latter
option, you must take reasonably prudent steps, when you begin
distribution of Opaque copies in quantity, to ensure that this
Transparent copy will remain thus accessible at the stated location
until at least one year after the last time you distribute an Opaque
copy (directly or through your agents or retailers) of that edition to
the public.

It is requested, but not required, that you contact the authors of the
Document well before redistributing any large number of copies, to give
them a chance to provide you with an updated version of the Document.
@sp 1
@item
MODIFICATIONS

You may copy and distribute a Modified Version of the Document under
the conditions of sections 2 and 3 above, provided that you release
the Modified Version under precisely this License, with the Modified
Version filling the role of the Document, thus licensing distribution
and modification of the Modified Version to whoever possesses a copy
of it.  In addition, you must do these things in the Modified Version:

A. Use in the Title Page (and on the covers, if any) a title distinct
   from that of the Document, and from those of previous versions
   (which should, if there were any, be listed in the History section
   of the Document).  You may use the same title as a previous version
   if the original publisher of that version gives permission.@*
B. List on the Title Page, as authors, one or more persons or entities
   responsible for authorship of the modifications in the Modified
   Version, together with at least five of the principal authors of the
   Document (all of its principal authors, if it has less than five).@*
C. State on the Title page the name of the publisher of the
   Modified Version, as the publisher.@*
D. Preserve all the copyright notices of the Document.@*
E. Add an appropriate copyright notice for your modifications
   adjacent to the other copyright notices.@*
F. Include, immediately after the copyright notices, a license notice
   giving the public permission to use the Modified Version under the
   terms of this License, in the form shown in the Addendum below.@*
G. Preserve in that license notice the full lists of Invariant Sections
   and required Cover Texts given in the Document's license notice.@*
H. Include an unaltered copy of this License.@*
I. Preserve the section entitled ``History'', and its title, and add to
   it an item stating at least the title, year, new authors, and
   publisher of the Modified Version as given on the Title Page.  If
   there is no section entitled ``History'' in the Document, create one
   stating the title, year, authors, and publisher of the Document as
   given on its Title Page, then add an item describing the Modified
   Version as stated in the previous sentence.@*
J. Preserve the network location, if any, given in the Document for
   public access to a Transparent copy of the Document, and likewise
   the network locations given in the Document for previous versions
   it was based on.  These may be placed in the ``History'' section.
   You may omit a network location for a work that was published at
   least four years before the Document itself, or if the original
   publisher of the version it refers to gives permission.@*
K. In any section entitled ``Acknowledgements'' or ``Dedications'',
   preserve the section's title, and preserve in the section all the
   substance and tone of each of the contributor acknowledgements
   and/or dedications given therein.@*
L. Preserve all the Invariant Sections of the Document,
   unaltered in their text and in their titles.  Section numbers
   or the equivalent are not considered part of the section titles.@*
M. Delete any section entitled ``Endorsements''.  Such a section
   may not be included in the Modified Version.@*
N. Do not retitle any existing section as ``Endorsements''
   or to conflict in title with any Invariant Section.@*
@sp 1
If the Modified Version includes new front-matter sections or
appendices that qualify as Secondary Sections and contain no material
copied from the Document, you may at your option designate some or all
of these sections as invariant.  To do this, add their titles to the
list of Invariant Sections in the Modified Version's license notice.
These titles must be distinct from any other section titles.

You may add a section entitled ``Endorsements'', provided it contains
nothing but endorsements of your Modified Version by various
parties--for example, statements of peer review or that the text has
been approved by an organization as the authoritative definition of a
standard.

You may add a passage of up to five words as a Front-Cover Text, and a
passage of up to 25 words as a Back-Cover Text, to the end of the list
of Cover Texts in the Modified Version.  Only one passage of
Front-Cover Text and one of Back-Cover Text may be added by (or
through arrangements made by) any one entity.  If the Document already
includes a cover text for the same cover, previously added by you or
by arrangement made by the same entity you are acting on behalf of,
you may not add another; but you may replace the old one, on explicit
permission from the previous publisher that added the old one.

The author(s) and publisher(s) of the Document do not by this License
give permission to use their names for publicity for or to assert or
imply endorsement of any Modified Version.
@sp 1
@item
COMBINING DOCUMENTS

You may combine the Document with other documents released under this
License, under the terms defined in section 4 above for modified
versions, provided that you include in the combination all of the
Invariant Sections of all of the original documents, unmodified, and
list them all as Invariant Sections of your combined work in its
license notice.

The combined work need only contain one copy of this License, and
multiple identical Invariant Sections may be replaced with a single
copy.  If there are multiple Invariant Sections with the same name but
different contents, make the title of each such section unique by
adding at the end of it, in parentheses, the name of the original
author or publisher of that section if known, or else a unique number.
Make the same adjustment to the section titles in the list of
Invariant Sections in the license notice of the combined work.

In the combination, you must combine any sections entitled ``History''
in the various original documents, forming one section entitled
``History''; likewise combine any sections entitled ``Acknowledgements'',
and any sections entitled ``Dedications''.  You must delete all sections
entitled ``Endorsements.''
@sp 1
@item
COLLECTIONS OF DOCUMENTS

You may make a collection consisting of the Document and other documents
released under this License, and replace the individual copies of this
License in the various documents with a single copy that is included in
the collection, provided that you follow the rules of this License for
verbatim copying of each of the documents in all other respects.

You may extract a single document from such a collection, and distribute
it individually under this License, provided you insert a copy of this
License into the extracted document, and follow this License in all
other respects regarding verbatim copying of that document.
@sp 1
@item
AGGREGATION WITH INDEPENDENT WORKS

A compilation of the Document or its derivatives with other separate
and independent documents or works, in or on a volume of a storage or
distribution medium, does not as a whole count as a Modified Version
of the Document, provided no compilation copyright is claimed for the
compilation.  Such a compilation is called an ``aggregate'', and this
License does not apply to the other self-contained works thus compiled
with the Document, on account of their being thus compiled, if they
are not themselves derivative works of the Document.

If the Cover Text requirement of section 3 is applicable to these
copies of the Document, then if the Document is less than one quarter
of the entire aggregate, the Document's Cover Texts may be placed on
covers that surround only the Document within the aggregate.
Otherwise they must appear on covers around the whole aggregate.
@sp 1
@item
TRANSLATION

Translation is considered a kind of modification, so you may
distribute translations of the Document under the terms of section 4.
Replacing Invariant Sections with translations requires special
permission from their copyright holders, but you may include
translations of some or all Invariant Sections in addition to the
original versions of these Invariant Sections.  You may include a
translation of this License provided that you also include the
original English version of this License.  In case of a disagreement
between the translation and the original English version of this
License, the original English version will prevail.
@sp 1
@item
TERMINATION

You may not copy, modify, sublicense, or distribute the Document except
as expressly provided for under this License.  Any other attempt to
copy, modify, sublicense or distribute the Document is void, and will
automatically terminate your rights under this License.  However,
parties who have received copies, or rights, from you under this
License will not have their licenses terminated so long as such
parties remain in full compliance.
@sp 1
@item
FUTURE REVISIONS OF THIS LICENSE

The Free Software Foundation may publish new, revised versions
of the GNU Free Documentation License from time to time.  Such new
versions will be similar in spirit to the present version, but may
differ in detail to address new problems or concerns.  See
http://www.gnu.org/copyleft/.

Each version of the License is given a distinguishing version number.
If the Document specifies that a particular numbered version of this
License ``or any later version'' applies to it, you have the option of
following the terms and conditions either of that specified version or
of any later version that has been published (not as a draft) by the
Free Software Foundation.  If the Document does not specify a version
number of this License, you may choose any version ever published (not
as a draft) by the Free Software Foundation.

@end enumerate

@unnumberedsec ADDENDUM: How to use this License for your documents

To use this License in a document you have written, include a copy of
the License in the document and put the following copyright and
license notices just after the title page:

@smallexample
@group

  Copyright (C)  @var{year}  @var{your name}.
  Permission is granted to copy, distribute and/or modify this document
  under the terms of the GNU Free Documentation License, Version 1.1
  or any later version published by the Free Software Foundation;
  with the Invariant Sections being @var{list their titles}, with the
  Front-Cover Texts being @var{list}, and with the Back-Cover Texts being @var{list}.
  A copy of the license is included in the section entitled ``GNU
  Free Documentation License''.
@end group
@end smallexample
If you have no Invariant Sections, write ``with no Invariant Sections''
instead of saying which ones are invariant.  If you have no
Front-Cover Texts, write ``no Front-Cover Texts'' instead of
``Front-Cover Texts being @var{list}''; likewise for Back-Cover Texts.

If your document contains nontrivial examples of program code, we
recommend releasing these examples in parallel under your choice of
free software license, such as the GNU General Public License,
to permit their use in free software.


@node Concept Index,  , Copying, Top
@unnumbered Concept Index
@printindex cp

@contents

@bye
